 \documentclass{openetcs_report}
% Use the option "nocc" if the document is not licensed under Creative Commons
%\documentclass[nocc]{template/openetcs_article}
\usepackage{todonotes}
\usepackage{appendix}
\usepackage{lipsum,url}
\usepackage{pdfpages}
%\usepackage{bibtopic} % Multibib
\usepackage{booktabs}
\usepackage{hyperref}

\usepackage[section,                 % add the glossary to the table of content 
            description,             % acronyms have a user-supplied description,
            style=superheaderborder, % table style
            nonumberlist             % no page number
]{glossaries}
\hypersetup{
linkbordercolor 	={1 1 1}}

%===========================
% Graphicpath
%===========================
\graphicspath{{./template/}{.}{./images/}}

%===========================
% Abbreviation file
%===========================
% \renewcommand*{\glossaryname}{List of Terms}
% \makeglossaries
% \loadglsentries{wp7_glossary} 
%===========================
%===========================
% Todo note margin
%===========================
\setlength{\marginparwidth}{7em}
%===========================

\begin{document}
\frontmatter
\project{openETCS}

%Please do not change anything above this line
%============================
% The document metadata is defined below

%assign a report number here
\reportnum{OETCS/WP7/D7.4}

%define your workpackage here
\wp{Work-Package 3: ``Tool chain''}

%set a title here
\title{openETCS:\\ openETCS Extended Tool Construction Set }

%set a subtitle here
\subtitle{Release 1.0}

%set the date of the report here
\date{October 2013}

%define a list of authors and their affiliation here

\author{Cecile Braunstein}

\affiliation{University Bremen}



% define the coverart
\coverart[width=350pt]{openETCS_EUPL}

%define the type of report
\reporttype{Installation guide}


\begin{abstract}
%define an abstract here
This document defines the first release of the openETCS tool chain. It
describes the installation procedure of the tool chain.
\end{abstract}

%=============================
%Do not change the next three lines
\maketitle
\tableofcontents
%\listoffiguresandtables

\newpage
%=============================

% The actual document starts below this line
%=============================
%Start here
%=============================
% Document Managment
%=============================
\chapter{Document Information}
\begin{tabular}{|p{4.4cm}|p{8.7cm}|}
\hline
\multicolumn{2}{|c|}{Document information} \\
\hline
Work Package &  WP7  \\
Deliverable ID or doc. ref. & D7.4\\
\hline
Document title & Tool chain release 1.0 \\
Document version & 01.00 \\
Document authors (org.)  & Cécile Braunstein  (Uni.Bremen)  \\
\hline
\end{tabular}

\begin{tabular}{|p{4.4cm}|p{8.7cm}|}
\hline
\multicolumn{2}{|c|}{Review information} \\
\hline
Last version reviewed &  \\
\hline
Main reviewers & \\
\hline
\end{tabular}

\begin{tabular}{|p{2.2cm}|p{4cm}|p{4cm}|p{2cm}|}
\hline
\multicolumn{4}{|c|}{Approbation} \\
\hline
  &  Name & Role & Date   \\
\hline  
Written by    &  Cécile Braunstein & WP7-T7.3 Sub-Task  & \\
&  & Leaders&\\
\hline
Approved by & &  & \\
\hline
\end{tabular}

\begin{tabular}{|p{2.2cm}|p{2cm}|p{3cm}|p{5cm}|}
\hline
\multicolumn{4}{|c|}{Document evolution} \\
\hline
Version &  Date & Author(s) & Justification  \\
\hline  
00.00 & 10.10.2013 & C. Braunstein  &  Document creation  \\
01.00 & 07.11.2013 & C. Braunstein  &  Review process preparation \\

\hline  
\end{tabular}
\newpage
%==========================================


%------ List of terms and definition ----------------
%\printglossary
%==========================================
\mainmatter
%----------------------
\chapter{openETCS release 1.0}
%----------------------

This release provides the tools for modeling at the system level and
for managing the requirements.


%----------------------
\section{List of Software}
Following the decision of WP7 deliverable D7.1 the first release of
the OpenETCS tool chain consists of the eclipse IDE with a set of plug-ins.

\begin{itemize}
\item Eclipse Modeling : Contains EMF, Egit, myLin, JDT, xsd ... 
\item Papyrus
\item ProR
\end{itemize}

%\todo[color=green!40,inline]{Short explanation and contains of each
%  components  ?}

%----------------------
\section{Installation Guide}
\subsection{Install Eclipse Modeling tools}
\begin{enumerate}
\item Download the
  \href{http://www.eclipse.org/downloads/packages/eclipse-modeling-tools/keplerr}{Eclipse
    Modeling}, choose the right version according to your Operating System.
\item Extract the contents 
\item Run eclipse
\end{enumerate}


 \subsection{Install \href{http://www.eclipse.org/papyrus/}{Papyrus}}
A more detailed installation guide may be found
\href{https://github.com/openETCS/toolchain/blob/master/ToolDescription/Papyrus/HowTo_install_papyrus_V1.2.0.pdf}{here}.
\begin{enumerate}
\item Click \verb+Help> Install Modeling Components+
\item Select Papyrus
\item confirm and accept the licence
\end{enumerate}


Guidelines for Papyrus are available on GitHub at this
\href{https://github.com/openETCS/model-evaluation/tree/master/model/Papyrus-CEA-All4tec/papyrus_guidelines}{location}

\subsection{Install \href{http://www.eclipse.org/rmf/pror/}{ProR}}
\begin{enumerate}
\item Click \verb+Help>Install new Software ...+
\item Add the new update site~:
\url{http://download.eclipse.org/rmf/updates} 
in the \verb+Work with:+ box
\item Select all items
\item Accept the terms of the license agreement
\item Finish the installation
\end{enumerate}
\section{Eclipse and  git repositories}
A tutorial on how to configure Egit may be found  \href{http://eclipsesource.com/blogs/tutorials/egit-tutorial/}{here}.
\subsection{How to import an Eclipse project from GitHub}
\begin{enumerate}
\item From Menu 
\verb+File>Import...+ select \verb+Git>Project from Git+
\item Choose "Clone URI"
\item Get the URL of your repository
\item Choose a protocol (e.g. https or ssh)
\item Choose the branch you want to clone and where the local copy
  will be set on your computer.
\item Click \verb+Import existing project+ and  select a project.

\end{enumerate}



%===================================================
%Do NOT change anything below this line

\end{document}
