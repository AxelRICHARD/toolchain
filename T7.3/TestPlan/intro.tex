\section{Introduction}
\label{sec:Introduction}


This document describes the way in which the Open ETCS platform will be tested in the validation stage, the test strategy covers, the OpenETCS interface and the plugins it works with.

\section{Executive Summary}
\label{sec-1-2}
The main objective is to ensure the requirements of the toolchain, are properly provided, such the user functionality and the interoperability between the different plugins involved in the platform.
The current document describes the strategy and objectives of the OpenETCS toolchain validation and details the validation test cases to be executed to the platform. The test cases have been design based on the functional specification and user’s guides provided into github.

\section{Intended Audience}
\label{sec-1-3}

The Tool chain Test Plan addresses all the stakeholders who are in the position to interact with OpenETCS tool chain.

\begin{itemize}
\item Project Manager
\item QA Manager
\item Assessors
\item Tool chain WP Leader
\item Tool chain Development Team
\item Tool chain Qualification Team
\item Tool chain V\&V team
\end{itemize}

\section{Evolution}

This document will be updated regularly with the evolution of the OpenETCS tool chain. The methods and tools to be applied during the development of the OpenETCS toolchain products will be decided based upon the results of the research activities carried out and the needed of the rest of WPs. 

The Tool Chain Test Plan document shall be updated whenever:
\begin{itemize}
\item tests or the approach for conducting them are changed
\item strategies or methodologies used in the Verification and Validation processes are modified
\item a new tool is added to the toolchain
\item a new tool or technique is incorporated in any of the tasks 
\item a tool is removed
\item a tool or a tool goal is modified
\end{itemize}

\section{References, Guidelines and Standards}

\begin{table}[htbp]
\begin{tabular}{|m{6,5cm}|m{4cm}|}
\hline
\multicolumn{2}{|c|}{\textbf{References}} \\\hline
Name &
Version/ Edition/ Date
\\\hline
Tool chain Development Plan & 00.03, 13.05.2014
\\\hline
Tool chain Qualification Process & 1.15, 21.10.2014
 \\\hline
\end{tabular}
\caption{References}
\end{table}

\section{Definitions and Abbreviations}

\begin{table}[htbp]
\begin{tabular}{|m{3cm}|m{8cm}|}
\hline
\textbf{Notation} &
\textbf{Description} 
\\\hline
MDT & Model Development Tools
 \\\hline
OBU & On Board Unit
\\\hline
QA & Quality Assurance
\\\hline
TC & Test Case
\\\hline
WP & Workpackage
\\\hline
\end{tabular}
\caption{Abbreviations}
\end{table}
%%% Local Variables: 
%%% mode: latex
%%% TeX-master: "TestPlan"
%%% End: 
