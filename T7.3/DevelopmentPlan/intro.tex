The purpose of this document is to set out the development plan
for the tool chain design, development and integration process.
Following the goals of the openETCS project defined in \cite{FPP},
this document contributes to the {\bf definition of a tool chain 
 for developing on board  software that can be certified by EN50128
 requirements (see \cite{standard_railway_2011})}.

The tool chain provides the tools support and the development process as
defined in \cite{D2.6} for the following openETCS activities :
\begin{itemize}
\item Formalization of the \gls{SSRS} 
\item Design  of the semi-formal model of \gls{ETCS} on board.
\item Design of the formal model of part of the \gls{ETCS} on board.
\item Code generation for  the \gls{OBU}
\item Documentation production
\item Requirements traceability 
\item Testing on various levels of the development process
\item Formal verification and  Validation of the produced software

\end{itemize}

The target software of the tool chain is an \gls{OBU} 
based on the \gls{SRS} Subset-026 \cite{unisig_subset-026_2012}.
The tool chain should help to assist the design process of {\em certifiable} \gls{SIL}4
software (see. \cite{D2.2}) for the \gls{OBU}. 

This document provides the following information:
\begin{itemize}
\item Methods for specifying the OpenETCS tool chain - section \ref{sec:toolchaindef}
\item Life cycle of the OpenETCS tool chain - section \ref{sec:lifecycle}
\item Development Environment of the OpenETCS tool chain - section \ref{sec:env}
\end{itemize}



%%% Local Variables: 
%%% mode: latex
%%% TeX-master: "WP7-ToolChainDevelpmentPlan.tex"
%%% End: 

