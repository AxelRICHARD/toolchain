\section{Tool Qualification}
\label{sec:sota-tool-qualification}


The CENELEC EN 50128 standard \cite{standard_railway_2011} defines the tool
qualification as follows:\\
{\it ``The objective is to provide evidence that potential
failures of tools do not adversely affect the integrated tool-set output in a
safety related manner that is undetected by technical and/or organizational
measures outside the tool. To this end, software tools are categorized into
three classes namely, T1, T2 \& T3 respectively.''}

We recall here the different class definitions:
\begin{itemize}
\item Tool class T1: No generated output can be used directly or indirectly to the
  safetey critical executable code;
\item Tool class T2: Verification tools, e.g. that can not introduce
  errors to the safety critical executable code but those tools may fail to detect errors or
  defects;
\item Tool class T3: Generated output directly or indirectly  as part of the
  safety critical executable code.
\end{itemize}
The deliverable D2.2 \cite{pokam_report_2013} summarizes the requirements for the
tool needed by the different tool classes. The report highlights that the
effort differs depending on the tool class. Furthermore, for
the most critical class T3,  the evidence should be provided that the output is
conform to the specification or that \emph{any failure in the output
  are detected}. 

The standard defined how to classify each tool individually (see
\cite{nielsen_efficient_2012,huang_test_2013} as an example).  But dealing with a tool
chain, integrated within a tool platform, implies extra effort to
ensure that the tool integration does not introduce new errors. For
example mechanism such as artifacts versioning, time-stamping
operations, etc ... should also be considered when qualifying the tool
chain. This increased the number of tools to consider during the
qualification process.

The effort of qualification depends on the number of tools under
consideration, the tool classes and the tool error detection
capabilities. To reduce the cost of the toolchain qualification and
regarding the fact that our development imply regular releases, a
systematic toolchain analysis approach has to be defined.

Table~\ref{tab:oetcs_tool_classification} exemplarily lists the tool
classes for some of the tools employed in the openETCS
project. Automatic code generator for SCADE model such as KCG or the
Bitwalker tool have a direct effect on the generated code whereas ProR
and Git do not. The classification for ProR would be different if it
would be used for formal requirements that then could be automatically
translated to a model (this would yield a T3-classification). The
verification tools RT-Tester and CPN Tools are not in the ``direct''
chain from requirements to code but they may fail to detect errors in
the software or model.

\begin{table}[h]
\begin{center}
\begin{tabular}{lll}\toprule
Tool&Purpose&Tool Class\\\midrule
Papyrus Editor &Definition of the model architecture&T1\\
Papyrus SysML checker & Check SysML conformity of the model &T2\\
SCADE Editor &Low-level modelling and code generation&T1\\
SCADE Code Generator & Code generation&T3\\
ProR&Requirements management&T1\\
Bitwalker&Generation of data structures for modelling&T3\\
Git&Versioning \& Traceability&T1\\
RTTester&Model-based testing&T2\\
CPN Tools&Model checking and test case generation&T2\\
\bottomrule
\end{tabular}
\end{center}
\caption{Classification of some tools in the openETCS tool chain}
\label{tab:oetcs_tool_classification}
\end{table}
 
\section{Toolchain Qualification State of the Art}
\label{sec-1-2}
Some recent works have been done in the field of toolchain
qualification from a variety of projects. The next section summarizes
the most significant ones.

\subsection{Slotosh and al. (project RECOMP)}
\label{sec:slotosh-approach}

 \cite{slotosch_model-based_2012} describes a model-based approach to tool
 qualification to comply with DO-330 and integration into the Eclipse
 development environment. The authors claim that the benefits of their
 method  are the following:

\begin{description}
\item[Clarity:] remove ambiguities;
\item[Re-usability and Transparency:] check for reuse in different toolchain;
\item[Completeness:] the model covers all parts of the  development
  process and tis traceable;
\item[Automation:] Some part of the process may be automated.
\end{description}

Their method is explained in detail in \cite{slotosch_iso_2012}: the
toolchain analysis is based on a domain specific toolchain model
they have defined. This model is used to represent the toolchain
structure as well as the tool confidence.  Their goal is to deduce the
tool confidence level and to expose specific qualification
requirements. Furthermore, their idea is not only to check tool by
tool but they follow a more holistic approach that makes use of
rearrangement and/or the extension of the toolchain to avoid the
certification of all tools. This allows them to reduce the
qualification effort by focusing only on the critical tools and making
use of already available information.  Moreover, checks with respect to inconsistencies,
such as missing descriptions, unused artifacts, etc., may be
automatically executed on the toolchain model. Finally, automated document generation is addressed.

In \cite{wildmoser_determining_2012} the application of tools and methods to an
industrial use case to determine the potential errors in the tool-chain is described.

\subsection{Asplund and al. (projects iFEST, MBAT)}
\label{sec:asplund-approach}

The authors investigate the question if there exists part of the environment related to tool
integration that may fall outside the tool qualification defined by the a norm
(ISO 26262 here \cite{asplund_qualifying_2012}). And if so, how tool integration
is affected by ensuring functional safety. One conclusion is that the tool
integration may lead to an increase of the qualification effort.

They also state that the standards (EN 50128, DO-178C and ISO26262)
are not sufficient to check safety of a toolchain, but some part of a
toolchain may be taken into account to mitigate the qualification
effort. 
They highlight 9 safety issues caused by tool integration that also
allow to be more exact when identifying software that have to be
qualified for certification purpose. 

They advocate to use take a ``system approach''  to deal with the qualification of tool integration within a toolchain. We should not
think about individual tools anymore.  Their system approach follows these steps:

\begin{enumerate}
\item Pre-qualification of development tools (requirements tools, design
  tools ...): provided by the vendors.
\item Pre-qualification at the tool-chain level: based on step 1 and
  reference work-flows; decomposition of higher level (project-wide) safety goals on tool level
\item Qualification at the toolchain level: check whether assumptions in step are fulfilled (use cases, environment, process) by the actual toolchain to be deployed.
\item Qualification at the tool level: based on the actual environment
  when deploying the toolchain.
\end{enumerate}
This approach leads them to separate the parts required for software tool
qualification and to identify safety issues related to tool integration.

In \cite{asplund_towards_2012}, they explore the step 2): identifying
the required safety goals due to tool integration and obtaining a
description of a reference work-flow and tool-chain with annotations regarding the mitigating effort.  They propose to use the TIL language, a
domain specific language for toolchain models.  The model of the tool
chain is used to perform a risk analysis and to annotate parts
that need mitigating effort for the safety issues due to tool
integration. 

\subsection{Biehl and al. (projects CESAR, iFEST, MBAT)}
\label{sec:biehl-approach}

Biehl proposed a Domain Specific Language named TIL for Generating Tool Integration
Solutions \cite{biehl_matthias_domain_2011}.  A toolchain is described in terms
of a number of ``Tool Adapters'' and the relation between them.
\begin{itemize}
\item Tool Adapters: expose data and functionality of a tool
\item Channels
  \begin{itemize}
  \item ControlChannel describes service calls
  \item DataChannel describes data exchanges
  \item TraceChannel  describes creation of a trace links
  \end{itemize}

\item Sequencer: describes sequential control flow (sequence of services)
\item User:  describes and limit the possible interaction
\item Repository:  provides storage and version management of tool data
\end{itemize}
This DSL allows early analysis of the toolchain.
It may generate part of tool adapter code based on the source and target
meta-model.

More recently, Biehl and al.  define a standard language for modeling
development processes as defined by OMG 2008. The language has been used in \cite{biehl_constructing_2012,biehl_early_2012} together with the TIL
language to tailor a toolchain following a process model. The goal is
to be able to model both the development process and the set of tools
used.  A process is defined as follows:
\begin{itemize}
\item Process: several Activities
\item Activity: set of linked Tasks, WorkProducts, Roles
\item A Role can perform a Task
\item A WorkProduct can be managed by a Tool
\item A Task can use a Tool
\end{itemize}
Using together the process development language  and the toolchain
language, in \cite{biehl_early_2012}, the authors  measure the alignment of a toolchain
with a product development process. The method proceeds as follows:
\begin{enumerate}
\item Inputs:
  \begin{itemize}
  \item formalized description of the toolchain design
  \item description of the process including the set of tools and their capabilities
  \end{itemize}
\item Initial verification graph
\item Automatic mapping links to the verification graph (acc. to mapping rules)
\item Apply alignment rule on the verification graph
\item Apply metrics to determine the degree of alignment between the tool-chain and the
   process
\end{enumerate}
The metrics and the misalignment list provide feedback to refine the tool-chain
design.



%%% Local Variables: 
%%% mode: latex
%%% TeX-master: "oetcs_qualification_process"
%%% End: 
