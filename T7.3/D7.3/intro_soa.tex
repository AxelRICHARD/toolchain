\section{Tool Qualification}
\label{sec-1-1}


The CENELEC EN 50128 standard \cite{standard_railway_2011} defines the tool
qualification as follows:\\
{\it ``The objective is to provide evidence that potential
failures of tools do not adversely affect the integrated tool-set output in a
safety related manner that is undetected by technical and/or organizational
measures outside the tool. To this end, software tools are categorized into
three classes namely, T1, T2 \& T3 respectively.''}

We recall here the different class definitions:
\begin{itemize}
\item Tool class T1: No generated output can be used directly or indirectly to the
  executable code;
\item Tool class T2: Verification tools, the tool may fail to detect errors or
  defects;
\item Tool class T3: Generated output directly or indirectly  as part of the
  executable code.
\end{itemize}
The deliverable D2.2 \cite{pokam_report_2013} summarizes the requirements for the
tool needed by the different tool classes. The report highlights that the
effort differs depending on the tool class. Furthermore, for
the most critical class T3,  the evidence should be provided that the output is
conform to the specification or that \emph{any failure in the output
  are detected}. 

The standard defined how to classify each tool individually (see
\cite{nielsen_efficient_2012,huang_test_2013} as an example).  But dealing with a tool
chain, integrated within a tool platform, implies extra effort to
ensure that the tool integration does not introduce new errors. For
example mechanism such as artifacts versioning, time-stamping
operations, etc ... should also be considered when qualifying the tool
chain.

In summary, the effort of qualification depends on the tool class and the tool
error detection capabilities. To reduce the cost of the toolchain qualification
and regarding the fact that our development imply regular releases, a systematic
toolchain analysis approach has to be defined.
 
\section{Toolchain Qualification State of the Art}
\label{sec-1-2}
Some recent works have been done in the field of toolchain
qualification from a variety of projects. The next section summarizes
the most significant ones.

\subsection{Slotosh and al. (project RECOMP)}
\label{sec-1-2.1}

 \cite{slotosch_model-based_2012} describes a model-based approach to tool
 qualification to comply with DO-330 and integration into the Eclipse
 development environment. The authors claim that the benefits of their
 method  are the following:

\begin{description}
\item[Clarity:] remove ambiguities;
\item[Re-usability and Transparency:] check for reuse in different toolchain;
\item[Completeness:] the model covers all parts of the  development
  process and tis traceable;
\item[Automation:] Some part of the process may be automated.
\end{description}

Their method is explained in detail in \cite{slotosch_iso_2012}: the
toolchain analysis is based on a domain specific toolchain model
they have defined. This model is used to represent the toolchain
structure as well as the tool confidence.  Their goal is to deduce the
tool confidence level and to expose specific qualification
requirements, furthermore, their idea is not only to check tool by
tool but to have a more holistic approach and makes use of
rearrangement and/or the extension of the toolchain to avoid the
certification of all tools. This allows them to reduce the
qualification effort by focusing only on the critical tools and make
use of already available information.  Moreover, some inconsistencies
check such as, missing description, unused artifacts ..., may be
automatically done by using the toolchain model. Finally, they
provide support to automatize the document production.

\cite{wildmoser_determining_2012} apply their tool and methods an
industrial use case to determine the potential errors in the tool-chain.

\subsection{Asplund and al. (projects iFEST, MBAT)}
\label{sec-1-2.2}

The authors investigate the question if there exists part of the environment related to tool
integration that may fall outside the tool qualification defined by the a norm
(ISO 26262 here \cite{asplund_qualifying_2012}). And if so, how tool integration
is affected by ensuring functional safety. One conclusion is that the tool
integration may lead to increase the qualification effort.

They also state that the standards (EN 50128, DO-178C and ISO26262)
are not sufficient to check safety of a toolchain, but some part of a
toolchain may be taken into account to mitigate the qualification
effort. 
They highlight 9 safety issues caused by tool integration that also
allow to be more exact when identifying software that have to be
qualified for certification purpose. 

They advocate that to deal with the qualification of tool integration
within a toolchain a system approach should be taken, we should not
thing about individual tools anymore.  Their proposed method is a ``System
Approach'' for toolchain qualification following these steps:

\begin{enumerate}
\item Pre-Qualification of development tool (requirements tools,design
  tools ...): provided by the vendors.
\item Pre-qualification at the tool-chain level: based on step 1 and
  reference work-flows, defined where are the safety critical part.
\item Qualification of the tool-chain: check differences of step 2 and
  the actual deployed toolchain.
\item Qualification at the tool level: based on the actual environment
  when deploying the toolchain.
\end{enumerate}
This approach leads them to separate the parts required to software tool
qualification and to identify safety issues related to tool integration.

In \cite{asplund_towards_2012}, they explore the step 2): identifying
the required safety goals due to tool integration and obtain a
description of a reference work-flow and tool-chain with annotation
about the mitigating effort.  They proposes to use the TIL language, a
domain specific language for toolchain model.  The model of the tool
chain is used to perform a risk analysis and to annotate parts
that need mitigating effort for the safety issues due to tool
integration. 

\subsection{Biehl and al. (projects CESAR, iFEST, MBAT)}
\label{sec-1-2.3}

Biehl proposed a Domain Specific Language named TIL for Generating Tool Integration
Solutions \cite{biehl_matthias_domain_2011}.  A toolchain is described in terms
of a number of ``Tool Adapters'' and the relation between them.
\begin{itemize}
\item ToolAdapters: exposes data and functionality of a tool
\item Channels
  \begin{itemize}
  \item ControlChannels describe service calls
  \item DataChannel describe data exchanges
  \item TraceChannel  describe creation of a trace links
  \end{itemize}

\item Sequencer: describe sequential control flow (sequence of services)
\item User:  describe and limit the possible interaction
\item Repository:  provide storage and version Management of tool data
\end{itemize}
This DSL allows early analysis of the toolchain.
It may generate part of tool adapter code based on the source and target
meta-model.

More recently, Biehl and al.  define a standard language for modeling
development process defines by OMG 2008. The language has been used in
\cite{biehl_constructing_2012,biehl_early_2012} together with the TIL
language to tailor a toolchain following a process model. The goal is
to be able to model both the development process and the set of tools
used.  A process is defined as follows:
\begin{itemize}
\item Process: several Activities
\item Activity: set of linked Tasks, WorkProducts, Roles
\item A Role can perform a Task
\item A WorkProduct can be anaged by a Tool
\item A Task can use a Tool
\end{itemize}
Using together the process development language  and the toolchain
language, in \cite{biehl_early_2012}, the authors  measure the alignment of a toolchain
with a product development process. The method proceeds as follows:
\begin{enumerate}
\item Inputs:
  \begin{itemize}
  \item formalized description of the toolchain design
  \item description of the process including the set of tools and their capabilities
  \end{itemize}
\item Initial verification graph
\item Automatic mapping links to the verification graph (acc. to mapping rules)
\item Apply alignment rule on the verification graph
\item Apply metrics to determine the degree of alignment btw the tool-chain and the
   process
\end{enumerate}
The metrics and the misalignment list provide feedback to refine the tool-chain
design.



%%% Local Variables: 
%%% mode: latex
%%% TeX-master: "oetcs_qualification_process"
%%% End: 
