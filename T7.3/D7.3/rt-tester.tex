\section{RT-tester qualification example}
\subsection{Tool identification}
\begin{description}
\item[Name:]  RTT-MBT - Model-Based Test Generator
\item[Version :]  8.9-4.1.2
\end{description}
\subsection{Tool Justification}
The use of RT-tester allow to tests the C code produced by SCADE,
B. It automatically generates tests for the following coverage
criteria:  basic control state coverage, transition coverage, MC/DC
coverage and requirement coverage.

\subsection{Use cases}
\paragraph{Tool classification}
\begin{description}
\item[Intended prurpose] Automated generation of test procedures for embedded HW/SW con-
trol systems
\item[Output] Test procedures to be executed in a test execution environment (both software
testing or hardware-in-the-loop testing environment) 
\item[Input] SysML test model (XMI format), test cases specification.
\item[Tool Class] T2: The tool may fail to detect errors or defects.
\end{description}
\paragraph{Use case description}
\begin{enumerate} 
\item {\bf Generation} Generate test cases, test data and test procedures from model.
\item {\bf Replay.} Replay test execution logs against the model.
\end{enumerate}

\paragraph{Potential Errors}
\begin{itemize}
\item Hazard 1: undetected SUT failures. The test oracles produced are
  incorrect and fail to observe a incorect behavior during test
  execution.
\item Hazard 2: undetected coverage failures. The test execution does
  not cover what it should. The test will pass but the result will not
  covered what it supposed to.
\end{itemize}


