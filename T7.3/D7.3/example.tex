\newpage
\section{Bitwalker Model-Based Qualification Example}
\label{sec:bitwalker-example}

This section gives an example of a tool analysis model as described in Section~\ref{sec:model-based-tool-quali} based on the specification import function of the Bitwalker tool.
\paragraph{Functions}

\begin{tabular}{|lp{0.7\textwidth}|}
\hline
\multicolumn{2}{|c|}{\bf FunctionBitwalkerSpecImport}\\
\bf Name& Bitwalker Specification Import\\
\bf Description&Parses the Subset 026 Word document and generates a the ``Data Dictionary'' representing data types, variables and messages\\
\bf Inputs&ArtifactSubset026-7, ArtifactSubset026-8\\
\bf Use Cases&UC1\\
\bf Outputs&ArtifactDataDictionary\\
\bf AnalysisElements&PotentialError1, PotentialError2, PotentialError3, PotentialError4, ErrorMitigation1, ErrorMitigation2, ErrorMitigation3\\
\hline
\end{tabular}

\paragraph{Artifacts}

\begin{tabular}{|lp{0.7\textwidth}|}
\hline
\multicolumn{2}{|c|}{\bf ArtifactSubset026-7}\\
\bf Name& Subset 026 Word Document\\
\bf Description&Subset 026-7 document containing the variable and data type definitions of the ETCS specification\\
\bf Format&MS Word\\
\hline
\end{tabular}


\begin{tabular}{|lp{0.7\textwidth}|}
\hline
\multicolumn{2}{|c|}{\bf ArtifactSubset026-8}\\
\bf Name& Subset 026 Word Document\\
\bf Description&Subset 026-8 document containing the message type definitions of the ETCS specification\\
\bf Format&MS Word\\
\hline
\end{tabular}

\begin{tabular}{|lp{0.7\textwidth}|}
\hline
\multicolumn{2}{|c|}{\bf ArtifactDataDictionary}\\
\bf Name& Data Dictionary\\
\bf Description&Data Dictionary representing data types, variables and messages\\
\bf Format&Papyrus SysML\\
\hline
\end{tabular}

\paragraph{Use Cases}

\begin{tabular}{|lp{0.7\textwidth}|}
\hline
\multicolumn{2}{|c|}{\bf UC1}\\
\bf Name& Subset 026 Transformation\\
\bf Description&Generation of a Papyrus data dictionary from the Subset 026-7 and 026-8 documents\\
\bf Actors&Papyrus Modeller\\
\bf Steps&\begin{enumerate} 
		\item Selection of ArtifactSubset026-8 input file
		\item Selection of ArtifactSubset026-9 input file
		\item Indicate target Data Dictionary
		\item Initiate transformation
	\end{enumerate}\\
\bf Success Condition&The Data Dictionary contains exactly all definitions from the input documents; there are no deviations\\
\hline
\end{tabular}

\paragraph{AnalysisElements}

\begin{tabular}{|lp{0.7\textwidth}|}
\hline
\multicolumn{2}{|c|}{\bf PotentialError1}\\
\bf Probability&High\\
\bf Description&Variable / message / type not detected or missing in output\\
\bf Mitigation&ErrorMitigation1\\
\hline
\end{tabular}

\begin{tabular}{|lp{0.7\textwidth}|}
\hline
\multicolumn{2}{|c|}{\bf ErrorMitigation1}\\
\bf Description&This error can possibly be detected and avoided as the modelling process is manual. Required in- or output described in the SRS can be added manually if missing.\\
\bf Error&PotentialError1\\
\hline
\end{tabular}

\begin{tabular}{|lp{0.7\textwidth}|}
\hline
\multicolumn{2}{|c|}{\bf PotentialError2}\\
\bf Probability&Medium\\
\bf Description&Variable or message have wrong type\\
\bf Comment&This is a very dangerous error, especially in the case of different precision integers or floats. The error may
                                       remain unnoticed and be propagated to the code leading to potentially fatal malfunctions.\\
\bf Mitigation&ErrorMitigation2\\
\hline
\end{tabular}

\begin{tabular}{|lp{0.7\textwidth}|}
\hline
\multicolumn{2}{|c|}{\bf ErrorMitigation2}\\
\bf Description&A mitigation is only possible by manual recheck (which would make an automatic conversion obsolete) of extensive/exhaustive testing or verification of the tool implementing the feature. However the inconsistent nature of the input document (Word) could prevent this.\\
\bf Error&PotentialError2\\
\hline
\end{tabular}

\begin{tabular}{|lp{0.7\textwidth}|}
\hline
\multicolumn{2}{|c|}{\bf PotentialError3}\\
\bf Probability&Medium\\
\bf Description&Missing fields in record / message\\
\bf Mitigation&ErrorMitigation3\\
\hline
\end{tabular}

\begin{tabular}{|lp{0.7\textwidth}|}
\hline
\multicolumn{2}{|c|}{\bf ErrorMitigation3}\\
\bf Description&Can be detected if the functionality of the field is described in the functional description (similar to ErrorMitigation1).\\
\bf Error&PotentialError3\\
\hline
\end{tabular}

\begin{tabular}{|lp{0.7\textwidth}|}
\hline
\multicolumn{2}{|c|}{\bf PotentialError4}\\
\bf Probability&Medium\\
\bf Description&Wrong naming of variable / message / type\\
\bf Mitigation&ErrorMitigation3\\
\hline
\end{tabular}

\paragraph{V\&V Activities}

\begin{tabular}{|lp{0.7\textwidth}|}
\hline
\multicolumn{2}{|c|}{\bf Correctness Inspection}\\
\bf Requirement&UC1/Success Condition\\
\bf Description&Manually inspect whether all definitions of ArtifactSubset026-7 and ArtifactSubset026-8 are actually represented correctly in the Data Dictionary\\
\hline
\end{tabular}



\textbf{Remark: } It will be necessary to provide evidence that critical errors such as PotentialError2 or PotentialError4 can be detected or are not present in the tool. 
                   Exhaustive verification will be difficult due to the unreliable structure of the input document. The result must be correct
                   also for ill-formed documents.

\section{RT-tester qualification example}
\subsection{Tool identification}
\begin{description}
\item[Name:]  RTT-MBT - Model-Based Test Generator
\item[Version :]  8.9-4.1.2
\end{description}
\subsection{Tool Justification}
The use of RT-tester allow to tests the C code produced by SCADE,
B. It automatically generates tests for the following coverage
criteria:  basic control state coverage, transition coverage, MC/DC
coverage and requirement coverage.

\subsection{Use cases}
\paragraph{Tool classification}
\begin{description}
\item[Intended prurpose] Automated generation of test procedures for embedded HW/SW con-
trol systems
\item[Output] Test procedures to be executed in a test execution environment (both software
testing or hardware-in-the-loop testing environment) 
\item[Input] SysML test model (XMI format), test cases specification.
\item[Tool Class] T2: The tool may fail to detect errors or defects.
\end{description}
\paragraph{Use case description}
\begin{enumerate} 
\item {\bf Generation} Generate test cases, test data and test procedures from model.
\item {\bf Replay.} Replay test execution logs against the model.
\end{enumerate}

\paragraph{Potential Errors}
\begin{itemize}
\item Hazard 1: undetected SUT failures. The test oracles produced are
  incorrect and fail to observe a incorect behavior during test
  execution.
\item Hazard 2: undetected coverage failures. The test execution does
  not cover what it should. The test will pass but the result will not
  covered what it supposed to.
\end{itemize}



