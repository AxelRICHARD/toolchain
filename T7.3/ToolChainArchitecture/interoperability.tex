To make the tool chain flexible, easy to maintained etc ... We need to
define the interface between the tools in a loosely way. The interoperability
mechanism should assist the flexibility as well as give some guidance
to integrate new tools along the tool chain development. 

We propose that for each tool chain activities the WP7 will agree on a
common input/output format. This will help to link the activities
altogether and ease the tools update or change.
It keeps the tools independent.

That implies to have an open interoperability specification.
The interoperability mechanism can be decomposed in different
concept.
\begin{itemize}
\item Architecture: what are the principles of the interoperability
  mechanism all tools agree on (e.g. RESTful architecture)
\item Communication : the communication protocol and data exchange
  format (e.g. XML)
\item Syntax: the structure of information represented in a common
  data format. What tools need to parse and interpret information
  (e.g. RDF)
\item Semantic layer : Basic semantic to make the tools understand
  each others. (e.g. Meta model EMF)
\end{itemize}

The OSLC project proposes a set of standard interface to link  the
different lifecycle activities together. The idea is to have a common
core protocol and architecture and then for each activities a
domain specific interface. All activities do not necessarily share the
same interest for the same data. Some part of the interfaces are
already covered, others still need to be specify. The amount of needed
work to comply to this specification should be quantified and a
decision of whether or not OpenETCS join the OSLC project should be
made.

During the tool chain development we need to maybe go step by step and
start with exchange by file and then add more restful architecture.
At each development step each tool should conform to the defined
interface.



%%% Local Variables: 
%%% mode: latex
%%% TeX-master: "WP7-Toolchain_architecture"
%%% End: 
