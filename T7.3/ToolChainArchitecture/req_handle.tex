\section{Activities purpose}
Requirement engineering is the starting point of the modeling
activities. It is the basis for describing how  all the actors and
tools may interact to produce the desired system. 

In OpenETCS the specification describing the requirement comes from
three document, two of them related to the \gls{ERTMS}
\cite{unisig_subset-026_2012,unisig_subset-034-_2012}, the third one
to the know-how of the operators and the different actors of the
design process. The translation into a requirement database in
computer readable format remains a manual task.

Figure \ref{fig:reqHand} presents different tools alternative to
realize the requirement management activities within our tool
chain. The activity may be achieved by combining some of these tools
to get the best of each. For example one can start by collecting the
requirements in a csv format and then add the dependency with ProR or
Papyrus.
\begin{figure}[htbp]
  \centering
  \includegraphics[width=\textwidth]{images/Req_Hand}
  \caption{Requirement Engineering}
  \label{fig:reqHand}
\end{figure}


\todo[inline]{Tools description or reference to their description}

\section{Input/Output artifacts}
\todo[inline]{Take a decision to the interface or exhibits the compatibility
  between the format and how it may be achieved}

%%% Local Variables: 
%%% mode: latex
%%% TeX-master: "toolchain_archi"
%%% End: 
