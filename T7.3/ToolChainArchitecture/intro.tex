

%-----------------------
\section{Motivation}
%-----------------------
One of the goals of the openETCS project is to "Provide a tool chain and process/methodologies for developing an on-board software that can fulfill the CENELEC requirements for SIL4 software".

The tool chain consists of activities to support producing certifiable software such as:
\begin{itemize}
\item Software planning           
\item Requirements tracing
\item Tool confidences
\item Documentation/report production
\item Testing
\item  Verification and validation
\end{itemize}
The tool chain also takes care of providing the following functioning infrastructure to allow robust distributed development within the defined life cycle.
\begin{itemize}
\item a continuous automated build system,
\item mechanisms to upgrade tools in the platform,
\item mechanisms to add tools to the chain at a later stage (without breaking compatibility),
\item modification and change control manager,
\item tool chain documentation system.
\end{itemize}
%-----------------------
\section{Scope of the document}
%-----------------------

%----------------------
 \section{Reference documents}
%----------------------- 
\begingroup
%\renewcommand{\section}[2]{}%
\renewcommand{\chapter}[2]{}% for other classes
  \bibliographystyle{plain}
  \bibliography{wp7_bibliography}
\endgroup
%%% Local Variables: 
%%% mode: latex
%%% TeX-master: WP7-Toolchain_architecture"
%%% End: 
