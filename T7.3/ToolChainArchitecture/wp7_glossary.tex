% new item : 
% \newglossaryentry{⟨label⟩}{⟨key-val list⟩}
%--------------------
% usage in latex file 
%--------------------
%     Preamble
%--------------------
% \usepackage[section, % add the glossary to the table of content 
%            description,% acronyms have a user-supplied description,
%            style=superheaderborder, % table style
%            nonumberlist      % no page number
%            ]{glossaries}
%\renewcommand*{\glossaryname}{List of Terms}
%\makeglossaries
%\loadglsentries{wp7_glossary} 
%--------------------
%     reference
%--------------------
% \gls{label}
%--------------------
% compile command
%--------------------
% makeglossaries
%--------------------
\newglossaryentry{SRS}{
name={SRS},
description={System Requirement Specification}
}
\newglossaryentry{SSRS}{
name={SSRS},
description={Subsystem Requirement Specification}
}
\newglossaryentry{ETCS}{
name={ETCS},
description={European Train Control System}
}
\newglossaryentry{OBU}{
name={OBU},
description={On Board Unit}
}
\newglossaryentry{SIL}{
name={SIL},
description={System Integrity Level}
}
\newglossaryentry{IDE}{
name={IDE},
description={Integrated Development Environment}
}
\newglossaryentry{OS}{
name={OS},
description={Operating System}
}
\newglossaryentry{GUI}{
name={GUI},
description={Graphical User Interface}
}

\newglossaryentry{EVC}{
name={EVC},
description={European Vital Computer}
}
\newglossaryentry{EMF}{
name={EMF},
description={Eclipse Modeling Framework}
}
\newglossaryentry{API}{
name={API},
description={Application programming inerface}
}
\newglossaryentry{ERTMS}{
name={ERTMS},
description={European rail traffic management system }
}




