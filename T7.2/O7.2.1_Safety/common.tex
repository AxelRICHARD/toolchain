\chapter{Template}
\label{sec:template}

\section{Instructions}

\begin{description}
\item[\textcolor{green}{Author}] Author of the approaches description  \todo{Name -  Company}
\item[\textcolor{blue}{Assessor 1}] First assessor of the approaches \todo{Name - Company}
\item[\textcolor{magenta}{Assessor 2}] Second assessor of the approaches \todo{Name - Company}
\end{description}

In the sequel, main text is under the responsibilities of the author.

\begin{author_comment}
Author can add comments using this format at any place.
\end{author_comment}

\begin{assessor1}
First assessor can add comments using this format at any place.
\end{assessor1}

\begin{assessor2}
Second assessor can add comments using this format at any place.
\end{assessor2}

When a note is required, please follow this list :
\begin{description}
\item[0] not recommended, not adapted, rejected
\item[1] weakly recommended, adapted after major improvements, weakly rejected
\item[2] recommended, adapted (with light improvements if necessary)  weakly accepted
\item[3] highly recommended, well adapted,strongly accepted
\item[*] difficult to evaluate with a note (please add a comment under the table)
\end{description}

All the notes can be commented under each table.

\section{Presentation}

This section gives a quick presentation of the approach and the tool.

\begin{description}
\item[Name] \todo{Name of the approach and the tool}
\item[Web site] \todo{if available, how to  find information}
\item[Licence] \todo{Kind of licence}
\end{description}

\paragraph{Abstract} Short abstract on the approach and tool (10 lines max)

\paragraph{Publications} Short list of publications on the approach (5 max)


\section{Common criteria on secondary means and tools}
\label{common}
This section discusses the common criteria of the means and tools according to the project requirements on tools and the results of T7.1.

\subsection{Project and WP2 requirements}

The objectives of this list of criteria is to check if the proposed means and tools meet the main criteria of the project: open-source approaches, usability, modularity, coverage of the objectives,...

According WP2 requirements, give a note for characteristics of the use of the tool (from 0 to 3) :

\begin{tabular}{|l | c | c | c | c|}
\hline
& \textcolor{green}{Author} & \textcolor{blue}{Assessor 1} & \textcolor{magenta}{Assessor 2} & Total \\
\hline 
Open Source (D2.6-02-074) & & & &  \\
\hline 
Portability to operating systems (D2.6-02-075) & & & &  \\
\hline
Cooperation of tools (D2.6-02-076) & & & &  \\
\hline
Robustness (D2.6-02-078) & & & & \\
\hline
Modularity (D2.6-02-078.1) & & & & \\
\hline
Documentation management (D2.6-02-078.02) & & & & \\
\hline
Distributed software development (D2.6-02-078.03)  & & & & \\
\hline
Simultaneous multi-users (D2.6-02-078.04)   & & & & \\
\hline
Issue tracking (D2.6-02-078.05) & & & & \\
\hline
Differences between models (D2.6-02-078.06) & & & & \\
\hline
Version management (D2.6-02-078.07) & & & & \\
\hline
Concurrent version development (D2.6-02-078.08) & & & & \\
\hline
Model-based version control (D2.6-02-078.09) & & & & \\
\hline
Role traceability (D2.6-02-078.10) & & & & \\
\hline
Safety version traceability (D2.6-02-078.11) & & & & \\
\hline
Model traceability (D2.6-02-079) & & & & \\
\hline
Tool chain integration & & & & \\
\hline
Scalability & & & & \\
\hline
User Friendliness & & & & \\
\hline
\end{tabular}



\subsection{Qualification}

This section discusses how the tool can be classified according EN50128 requirements (D2.6-02-085). Some qualification shall be mandatory  if the tool is involved to design a SIL4 software.


\begin{tabular}{|l | c | c | c | c|}
\hline
& \textcolor{green}{Author} & \textcolor{blue}{Assessor 1} & \textcolor{magenta}{Assessor 2} & Total \\
\hline 
Tool manual (D.2.6-01-42.02) & & & &  \\
\hline
Proof of correctness (D.2.6-01-42.03)   & & & & \\
\hline
Existing industrial  usage  & & & & \\
\hline
Model verification & & & & \\
\hline
Test generation & & & & \\
\hline
Simulation, execution, debugging & & & & \\
\hline
Formal proof & & & & \\
\hline
\end{tabular}


Which scope of qualification is expected according EN50128 (section 6.7) ?

Score :
\begin{description}
\item[3] already qualified for this level
\item[2] qualification possible to this level, but some elements shall be provided
\item[0] qualification not recommended for this level
\end{description}


\begin{tabular}{|l | c | c | c | c|}
\hline
& \textcolor{green}{Author} & \textcolor{blue}{Assessor 1} & \textcolor{magenta}{Assessor 2} & Total \\
\hline 
class T1 & & & &  \\
\hline
class T2   & & & & \\
\hline
class T3  & & & & \\
\hline
\end{tabular}

\paragraph{Other elements for tool certification}


\subsection{Complementarity with primary toolchain}

The objectives of this list of criteria is to check if the proposed means and tools can be easily integrated to the primary toolchain.

\subsubsection{Language}


According to the decisions and the propositions of T7.1, how the mean and approach can be adapted to or can complete the chosen language and methods:

\begin{tabular}{|l | c | c | c | c|}
\hline
& \textcolor{green}{Author} & \textcolor{blue}{Assessor 1} & \textcolor{magenta}{Assessor 2} & Total \\
\hline 
SysML  & & & & \\
\hline
Scade method & & & & \\
\hline
EFS language & & & & \\
\hline
B Method & & & & \\
\hline
C language & & & & \\
\hline
\end{tabular}

\paragraph{SysML}
How the means or tools can complete SysML ?


\paragraph{Scade, EFS, Classical B}
How the means or tools can complete the current proposals for formal modeling language ?

\paragraph{C language}
How the means or tools can complete or be adapted to SIL4 software in C language ?

\subsubsection{Tools and platforms}

According to the decisions and the propositions of T7.1, how the mean and approach can be integrated to or can complete the chosen tools and platforms:

\begin{tabular}{|l | c | c | c | c|}
\hline
& \textcolor{green}{Author} & \textcolor{blue}{Assessor 1} & \textcolor{magenta}{Assessor 2} & Total \\
\hline 
Eclipse & & & &  \\
\hline
Papyrus  & & & & \\
\hline
Scade & & & & \\
\hline
EFS tools & & & & \\
\hline
B tools & & & & \\
\hline
\end{tabular}


\paragraph{Eclipse}
How the means or tools can be integrated to the Eclipse platform ?

\paragraph{Papyrus}
How the means or tools can complete  Papyrus ?


\paragraph{Scade, EFS, Classical B}
How the means or tools can complete the current proposals for formal modeling tools ?


\section{Means and tools for safety activities support}
\label{sec:safety}


This section defines the criteria for the means and tools dedicated to support of safety activities, in the WP4 workpackage. 

Criteria of this section are defined according \citep{D4.2.a}.

\subsection{Safety activities}

Which safety design activities are covered by the mean or tool (see \citep{D4.2.a} section 1.2) ?

\begin{tabular}{|l | c | c | c | c|}
\hline
& \textcolor{green}{Author} & \textcolor{blue}{Assessor 1} & \textcolor{magenta}{Assessor 2} & Total \\
\hline 
Preliminary Hazard Analysis & & & &  \\
\hline
System Hazard and Risk Analysis & & & & \\
\hline
Risk Assessment & & & & \\
\hline
Specification of System Safety Requirements & & & &  \\
\hline
Define Safety Related Functional Requirements & & & & \\
\hline
Specify Sub-System and Component & & & & \\
Safety requirements & & & & \\
\hline
Verify System, Sub-System and Component & & & &  \\
Safety requirements & & & &  \\
\hline
Validate System Safety Requirements & & & & \\
\hline
Establish Safety Case & & & & \\
\hline
\end{tabular}


\subsection{Input Artifacts}

Which artifacts are used as input of the mean or tool (see \citep{D4.2.a} section 1.4) ? 


\begin{tabular}{|l | c | c | c | c|}
\hline
& \textcolor{green}{Author} & \textcolor{blue}{Assessor 1} & \textcolor{magenta}{Assessor 2} & Total \\
\hline 
Safety Requirement & & & &  \\
\hline
Hazard log & & & & \\
\hline
Safety Case & & & & \\
\hline
\end{tabular}



\subsection{Output Artifacts}

Which artifacts are used as output of the mean or tool (see \citep{D4.2.a} section 1.4) ? 


\begin{tabular}{|l | c | c | c | c|}
\hline
& \textcolor{green}{Author} & \textcolor{blue}{Assessor 1} & \textcolor{magenta}{Assessor 2} & Total \\
\hline 
Safety Requirement & & & &  \\
\hline
Hazard log & & & & \\
\hline
Safety Case & & & & \\
\hline
\end{tabular}

\subsection{Expressiveness}


Which degree of formalisation is given to the artifacts by mean or tools (see \citep{D4.2.a} section 1.4) ? 


\begin{tabular}{|l | c | c | c | c|}
\hline
& \textcolor{green}{Author} & \textcolor{blue}{Assessor 1} & \textcolor{magenta}{Assessor 2} & Total \\
\hline 
Informal & & & &  \\
\hline
Semi-Formal & & & & \\
\hline
Formal & & & & \\
\hline
\end{tabular}


\subsection{Other criteria}
According to \citep{D4.2.a} section 2.2, provide some complement on the mean or tool:


\begin{tabular}{|l | c | c | c | c|}
\hline
& \textcolor{green}{Author} & \textcolor{blue}{Assessor 1} & \textcolor{magenta}{Assessor 2} & Total \\
\hline 
Top-Down approach  & & & &  \\
\hline
Bottom-up approach & & & & \\
\hline
Database capability & & & & \\
\hline
Database query ability & & & & \\
\hline
Safety requirement VnV & & & & \\
\hline
Traceability & & & & \\
\hline
Generation of documentation & & & & \\
\hline
\end{tabular}


\section{Other comments}



\begin{comment}
This section is available for the author or the assessors to  complete the description and criteria.
\end{comment}



