\documentclass{beamer}

% This file is a solution template for:

% - Talk at a conference/colloquium.
% - Talk length is about 20min.


% Copyright 2004 by Till Tantau <tantau@users.sourceforge.net>.
%
% In principle, this file can be redistributed and/or modified under
% the terms of the GNU Public License, version 2.
%
% However, this file is supposed to be a template to be modified
% for your own needs. For this reason, if you use this file as a
% template and not specifically distribute it as part of a another
% package/program, I grant the extra permission to freely copy and
% modify this file as you see fit and even to delete this copyright
% notice. 


\mode<presentation>
{
  \usetheme{openETCS}
  % or ...

  \setbeamercovered{transparent}
  % or whatever (possibly just delete it)
  \setbeamertemplate{navigation symbols}{}

}


\usepackage[english]{babel}
% or whatever

\usepackage[utf8]{inputenc}
% or whatever

\usepackage{times}
\usepackage[T1]{fontenc}
% Or whatever. Note that the encoding and the font should match. If T1
% does not look nice, try deleting the line with the fontenc.


\title[Progress meeting T7.2] % (optional, use only with long paper titles)
{Progress meeting on secondary tools}

\subtitle
{WP7 Task 7.2}

\author[WP7 - Task 7.2] % (optional, use only with lots of authors)
{Marielle Petit-Doche}
% - Give the names in the same order as the appear in the paper.
% - Use the \inst{?} command only if the authors have different
%   affiliation.

\institute[Systerel] % (optional, but mostly needed)
{\pgfdeclareimage[height=1cm]{systerel-logo}{logoSysterel}
 \pgfuseimage{systerel-logo}
 }
% - Use the \inst command only if there are several affiliations.
% - Keep it simple, no one is interested in your street address.

\date[4th of July, Paris] % (optional, should be abbreviation of conference name)
{4th of July 2013, UIC, Paris}
% - Either use conference name or its abbreviation.
% - Not really informative to the audience, more for people (including
%   yourself) who are reading the slides online

\subject{OpenETCS WP7-T7.2 Progress Meeting}
% This is only inserted into the PDF information catalog. Can be left
% out. 


% If you have a file called "university-logo-filename.xxx", where xxx
% is a graphic format that can be processed by latex or pdflatex,
% resp., then you can add a logo as follows:

%\pgfdeclareimage[height=0.5cm]{university-logo}{logo-uni}
%\logoinst{\pgfuseimage{university-logo}}



% Delete this, if you do not want the table of contents to pop up at
% the beginning of each subsection:
\AtBeginSubsection[]
{
  \begin{frame}<beamer>{Outline}
    \tableofcontents[currentsection,currentsubsection]
  \end{frame}
}


% If you wish to uncover everything in a step-wise fashion, uncomment
% the following command: 

%\beamerdefaultoverlayspecification{<+->}


\begin{document}

\begin{frame}
  \titlepage
\end{frame}

\begin{frame}{Outline}
  \tableofcontents
  % You might wish to add the option [pausesections]
\end{frame}


% Structuring a talk is a difficult task and the following structure
% may not be suitable. Here are some rules that apply for this
% solution: 

% - Exactly two or three sections (other than the summary).
% - At *most* three subsections per section.
% - Talk about 30s to 2min per frame. So there should be between about
%   15 and 30 frames, all told.

% - A conference audience is likely to know very little of what you
%   are going to talk about. So *simplify*!
% - In a 20min talk, getting the main ideas across is hard
%   enough. Leave out details, even if it means being less precise than
%   you think necessary.
% - If you omit details that are vital to the proof/implementation,
%   just say so once. Everybody will be happy with that.



\section{SSRS needs}


\begin{frame}{Aim}

  \begin{itemize}
  \item
    Identify the functional architecture of the system (graphical view ?)
  \item
    Manage a set of function
  \item 
  	Manage a repository of requirements
  \item 
  	Manage a data dictionnary
  \end{itemize}

\end{frame}

\section{Expected inputs}

\begin{frame}{Next step}

  \begin{itemize}
  \item
    Specify information needed to describe function
  \item
    Specify information needed to  describe requirement
  \item 
  	Specify information needed to describe data
  \item 
  	Specify links between function, requirement, data
  \end{itemize}
  
  \textcolor{red}{$ \Rightarrow $ Urgent} to have a precise specification of the minimal needs:  Next week

\end{frame}


\section{WP7-T7.2 Concerns}


\begin{frame}{Requirements on means to support activities}

  \begin{itemize}
  \item
    Possibilities to share between users
  \item
    Possibilities to share information between activities
  \item 
  	Flexibility : possibility to add new elements to describes the items
  \item
  Open source
  \item 
  	\dots
  \end{itemize}
  
\end{frame}



\begin{frame}{Initial solutions ?}

  \begin{itemize}
  \item
    Pen and paper
  \item
    Excel sheet
  \item 
  	SysML
  \item
  	Plug-ins on eclipse (ProR,...)
  \item 
  	\dots
  \end{itemize}
  
\end{frame}


\section{Organisation}



\begin{frame}{T7.2 Organisation}

\begin{description}
\item[Inputs] Data model needed by end of July
\item[Benchmark] Analyse of needs and possible solution of support
\item[Decision] With other secondary tools :  end of September
\end{description}
  
\end{frame}


\section{T7.1 Conclusion}


\begin{frame}{T7.1 Proposition}



\begin{itemize}
\item More than a year for the evaluation (1/3 of project duration)
\item Now at the middle term of the project
\item Two decision meetings with votes
\item Few activities mentioned as follow-up of the meetings
\item Next steps started in the project (WP7, WP3, WP4,...)
\item Few resources available
\end{itemize}

  \pause
  \textcolor{red}{Can we consider officially this task achieved ?}

\end{frame}



\end{document}


