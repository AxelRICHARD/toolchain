

\chapter{Verification and Validation}
\label{sec:VnV}

This section is dedicated to tools and means for verification and validation.

\section{Candidates}


The list of initial candidates is:

\begin{itemize}
\item Scade Suite
\item System C
\item UPPAAL
\item Rodin and Pluggins
\item Tools around Classical B (ProB, SMT solver,...)
\item CPN tools
\item Matelo
\item RT-Tester
\item Fiacre and Tina
\item Frama-C
\item Diversity
\item SPIN
\end{itemize}

\subsection{SystemC}

\begin{description}
\item[Name] SystemC
\item[Web site] \url{www.accellera.org/downloads/standards/systemc/about_systemc/}
\item[Licence] SystemC Open Source License
\end{description}

\paragraph{Abstract} SystemC is a C++ library providing an event-driven simulation interface suitable for electronic system level design. It enables a system designer to simulate concurrent processes. SystemC processes can communicate in a simulated real-time environment, using channels of different datatypes (all C++ types and user defined types are supported). SystemC supports hardware and software synthesis (with the corresponding tools). SystemC models are executable.

\paragraph{Publications} 

\begin{itemize}
\item D. C. Black, SystemC: From the ground up. Springer, 2010.
\item IEEE 1666 Standard SystemC Language Reference Manual, \url{http://standards.ieee.org/getieee/1666/}
\item The ITEA MARTES Project, from UML to SystemC, \url{http://www.martes-itea.org/}
\item J. Bhasker, A SystemC Primer, Second Edition, Star Galaxy Publishing, 2004
\item F. Ghenassia (Editor), Transaction-Level Modeling with SystemC: TLM Concepts and
Applications for Embedded Systems, Springer 2006
\end{itemize}


A quantitative evaluation is available in \url{https://github.com/openETCS/toolchain/blob/master/T7.2/O7.2.1_VnV/O7-2-1_VnV.pdf}

\paragraph{Added value for OpenETCS project}

\begin{comment}
To complete: Stefan Rieger  ?
\end{comment}


\paragraph{Integration in OpenETCS process and toolchain}

\begin{comment}
To complete: Stefan Rieger  ?
\end{comment}


\subsection{UPPAAL}

\begin{description}
\item[Name] UPPAAL
\item[Web site] www.uppaal.org
\item[Licence] Academic free or commercial license
\end{description}

\paragraph{Abstract} Uppaal is an integrated tool environment for modeling, validation and verification of real-time systems modeled as networks of timed automata, extended with data types (bounded integers, arrays, etc.).

\paragraph{Publications} Short list of publications on the approach (5 max)
Please refer to \verb|http://dblp.org/search/#query=uppaal|



A quantitative evaluation is available in \url{https://github.com/openETCS/toolchain/blob/master/T7.2/O7.2.1_VnV/O7-2-1_VnV.pdf}


\paragraph{Added value for OpenETCS project}

\begin{comment}
To complete: Stefan Rieger  ?
\end{comment}


\paragraph{Integration in OpenETCS process and toolchain}

\begin{comment}
To complete: Stefan Rieger  ?
\end{comment}


\subsection{CPN Tools}



\begin{description}
\item[Name] CPN Tools
\item[Website] http://cpntools.org/
\item[Licence] Open Source (GPL/LGPL)
\end{description}

\paragraph{Abstract} CPN Tools is a tool for editing, simulating, and analyzing Colored Petri nets.

The tool features incremental syntax checking and code generation, which take place while a net is being constructed. A fast simulator efficiently handles untimed and timed nets. Full and partial state spaces can be generated and analyzed, and a standard state space report contains information, such as boundedness properties and liveness properties.

\paragraph{Publications} Please refer to http://cpntools.org/publications


Slides available on github \url{https://github.com/openETCS/model-evaluation/blob/master/Telco_Secondary_slides/b-Introduction_CPNTools.pdf}.

A quantitative evaluation is available in \url{https://github.com/openETCS/toolchain/blob/master/T7.2/O7.2.1_VnV/O7-2-1_VnV.pdf}

\paragraph{Added value for OpenETCS project}

\begin{comment}
To complete: Stefan Rieger , Jan Welte ?
\end{comment}


\paragraph{Integration in OpenETCS process and toolchain}

\begin{comment}
To complete: Stefan Rieger , Jan Welte ?
\end{comment}

\section{Selected means and tools}

\begin{comment}
To complete after decision meeting with a section for each tool with the following contents:

\begin{itemize}
\item description of the means or tools, references and links
\item added value for openETCS
\item for which tasks and how (input/output/actions) is the mean or tools used.
\end{itemize}
\end{comment}