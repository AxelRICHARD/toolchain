

\chapter{Verification and Validation}
\label{sec:VnV}

This section is dedicated to tools and means for verification and validation.

\section{Candidates}


The list of initial candidates is:

\begin{itemize}
\item Scade Suite
\item System C
\item UPPAAL
\item Rodin and Pluggins
\item Tools around Classical B (ProB, SMT solver,...)
\item CPN tools
\item Matelo
\item RT-Tester
\item Fiacre and Tina
\item Frama-C
\item Diversity
\item SPIN
\end{itemize}

\subsection{SystemC}

\begin{description}
\item[Name] SystemC
\item[Web site] \url{www.accellera.org/downloads/standards/systemc/about_systemc/}
\item[Licence] SystemC Open Source License
\end{description}

\paragraph{Abstract} SystemC is a C++ library providing an event-driven simulation interface suitable for electronic system level design. It enables a system designer to simulate concurrent processes. SystemC processes can communicate in a simulated real-time environment, using channels of different datatypes (all C++ types and user defined types are supported). SystemC supports hardware and software synthesis (with the corresponding tools). SystemC models are executable.

\paragraph{Publications} 

\begin{itemize}
\item D. C. Black, SystemC: From the ground up. Springer, 2010.
\item IEEE 1666 Standard SystemC Language Reference Manual, \url{http://standards.ieee.org/getieee/1666/}
\item The ITEA MARTES Project, from UML to SystemC, \url{http://www.martes-itea.org/}
\item J. Bhasker, A SystemC Primer, Second Edition, Star Galaxy Publishing, 2004
\item F. Ghenassia (Editor), Transaction-Level Modeling with SystemC: TLM Concepts and
Applications for Embedded Systems, Springer 2006
\end{itemize}


A quantitative evaluation is available in \url{https://github.com/openETCS/toolchain/blob/master/T7.2/O7.2.1_VnV/O7-2-1_VnV.pdf}

\paragraph{Added value for OpenETCS project}

\begin{comment}
To complete: Stefan Rieger  ?
\end{comment}


\paragraph{Integration in OpenETCS process and toolchain}

\begin{comment}
To complete: Stefan Rieger  ?
\end{comment}


\subsection{UPPAAL}

\begin{description}
\item[Name] UPPAAL
\item[Web site] www.uppaal.org
\item[Licence] Academic free or commercial license
\end{description}

\paragraph{Abstract} Uppaal is an integrated tool environment for modeling, validation and verification of real-time systems modeled as networks of timed automata, extended with data types (bounded integers, arrays, etc.).

\paragraph{Publications} Short list of publications on the approach (5 max)
Please refer to \verb|http://dblp.org/search/#query=uppaal|



A quantitative evaluation is available in \url{https://github.com/openETCS/toolchain/blob/master/T7.2/O7.2.1_VnV/O7-2-1_VnV.pdf}


\paragraph{Added value for OpenETCS project}

\begin{comment}
To complete: Stefan Rieger  ?
\end{comment}


\paragraph{Integration in OpenETCS process and toolchain}

\begin{comment}
To complete: Stefan Rieger  ?
\end{comment}


\subsection{CPN Tools}

\begin{description}
\item[Name] CPN Tools
\item[Website] http://cpntools.org/
\item[Licence] Open Source (GPL/LGPL)
\end{description}

\paragraph{Abstract} CPN Tools is a tool for editing, simulating, and analyzing Colored Petri nets.

The tool features incremental syntax checking and code generation, which take place while a net is being constructed. A fast simulator efficiently handles untimed and timed nets. Full and partial state spaces can be generated and analyzed, and a standard state space report contains information, such as boundedness properties and liveness properties.

\paragraph{Publications} Please refer to http://cpntools.org/publications

Slides available on github \url{https://github.com/openETCS/model-evaluation/blob/master/Telco_Secondary_slides/b-Introduction_CPNTools.pdf}.

A quantitative evaluation is available in \url{https://github.com/openETCS/toolchain/blob/master/T7.2/O7.2.1_VnV/O7-2-1_VnV.pdf}

\paragraph{Added value for OpenETCS project}

Petri Nets as a means of description are used in research an in industrial applications used for Process Modeling, Data analysis, Software design and Reliability engineering. Coloured Petri Nets are  High-level Petri Nets which are mainly used to describe, simulate and validated communication between humans and/or computers. As a means of description Coloured and  Hierarchic Petri nets allow to use one uniform means of description for the entire development cycle, starting with  the specification through to  implementation. 
Coloured petri nets are standardised as part of the high level petri nets in ISO/IEC 15909 Systems and software engineering - High-level Petri nets. The use of petri nets for the system dependability analysis is standardised in IEC 62551 Analysis techniques for dependability - Petri net modeling. In addition Coloured petri nets and the CPN Tools are introduced and documented in the book \textit{Coloured Petri Nets -- Modeling and Validation of Concurrent Systems} by K. Jensen and L.M. Kristensen. CPN Tools is a mature tool suite for coloured petri nets which provides support to edit, check, simulate and analyse nets on all relevant abstraction levels. CPN Tools has a graphical editor to model nets and provides various methods to analyse the nets, most importantly a reachability analysis. 

Petri nets are a strictly formal means of description suited for formal proof of behavioural and structural properties. The nets are mainly verified by generation and analysis of the state space. The tool supports the calculation and drawing of the state space, which is used to verify certain logical and temporal properties of the system. Additional model checker can be combined with the tool to provide additional functionalities. In context of the openETCS work coloured petri nets and CPN Tools provide a variable and efficient option for test models to validate the behaviour of the openETCS model. Especially, CPN Tools models can be used to specify and test timing properties.

The simulation engine of CPN tools provides a powerful simulation of petri nets and has a number of debugging functions. It does not support test generation, but provides interfaces for other tools to do so. Correspondingly, tools like SPENAT can be used to generate and manage all kinds of tests for the nets created with CPN Tools. 

\paragraph{Integration in OpenETCS process and toolchain}

CPN Tools is a mainly open source and for free tool suite which provides a number of interfaces to interact with other modeling tools and simulations. Depending on the specific use of CPN Tools to validate model behaviour or specifications these interfaces can be used co-simulated with SysML or SCADE models to run test cases. CPN Tools can also be as an independent to to just build test models and derive a number of test cases. 

%------------------------------- RT-Tester
\subsection{RT-Tester}

\begin{description}
\item[Name] RT-Tester
\item[Web site] http://www.verified.de/en/products/rt-tester
\item[Licence] 
  \begin{itemize}
    \item Generator:  closed
    \item SMT Solver: open
    \item GUI (eclipse plug-in): open
  \end{itemize}
\end{description}

\paragraph{Abstract}
The RT-Tester test automation tool, made by Verified, performs
automatic test generation, test execution and real-time test
evaluation.  It supports different testing approach such as unit
testing, software integration testing for component, hardware/software
integration testing and system integration testing.  
The tool generates automatically behavioral tests that covers the
classical criteria such as  reachable state, branch and MC/DC.
The RT-Tester also implements the so-called  model-based testing approach: 
starting from a test model design with UML/SysML, the RT-tester fully
automatically generates test cases. Moreover the test model may be
directly linked to the requirements, thus the requirement coverage may
also be ensured.

The RT-tester provides the following features :
\begin{itemize}
\item Automated Test Case Generation 
\item Automated Test Data and Test Oracles Generation 
\item Automated Test Procedure Generation 
\item Automated Requirement Tracing 
\item Test Management system 
\item Test Report Generation
\end{itemize}


Note that the RT-tester model based testing has be qualified according
to ISO 26262.

\paragraph{Publications}

\begin{itemize}
\item Jan Peleska, Elena Vorobev, and Florian Lapschies. Automated test case generation with smt-solving
and abstract interpretation. In Mihaela Bobaru, Klaus Havelund, GerardJ. Holzmann, and Rajeev
Joshi, editors, NASA Formal Methods, volume 6617 of Lecture Notes in Computer Science, pages
298–312. Springer Berlin Heidelberg, 2011.
\item Jan Peleska, Elena Vorobev, Florian Lapschies, and Cornelia Zahlten. Automated model-based
testing with RT-Tester. Technical report, Universität Bremen, 2011.
\item Jörg Brauer, Jan Peleska, and Uwe Schulze. Efficient and trustworthy tool qualification for model-
based testing tools. In Brian Nielsen and Carsten Weise, editors, Testing Software and Systems,
volume 7641 of Lecture Notes in Computer Science, pages 8–23. Springer Berlin Heidelberg, 2012.
\item Verified Systems International GmbH. Verified :: Products. http://www.verified.de/en/products.


\end{itemize} 

\paragraph{Added value for OpenETCS project}

RT-Tester will add  an automated test case and test data generator
that interprets test models directly describes in SysML. 
The test models may specify concurrent sub-components of
SUT (System Under Test) and TE (Test Environment), and timing
conditions using dense time (i. e., continuous physical time) and 
an arbitrary number of timers. 

Another added value is that the tool also provides a simulation
environment that can directly generate C code from the SysML model and
run the tests.
The code generator may also be adapted to other purposes.

\paragraph{Integration in OpenETCS process and toolchain}

\begin{itemize}
  \item GUI via an eclipse plug-in already exist,
  \item read SysML model
  \item Can run SCADE generated code
\end{itemize}

%---------------------------------------------------------------------

\section{Selected means and tools}

\begin{comment}
To complete after decision meeting with a section for each tool with the following contents:

\begin{itemize}
\item description of the means or tools, references and links
\item added value for openETCS
\item for which tasks and how (input/output/actions) is the mean or tools used.
\end{itemize}
\end{comment}