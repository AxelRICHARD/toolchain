

\chapter{Data and Requirements Management}
\label{sec:management}

 
This section is dedicated to tools and means to support management of data, functions requirements and other artifacts along  the openETCS process.

In total, seven tools have been proposed.  Out of these, only one has been evaluated in detail (ProR).  What follows is a qualitative description of the seven tools.  A quantitative evaluation of ProR is included as well.

\section{Candidates}
The list of initial candidates is:

\begin{description}
\item[Scade Suite.]  Scade includes Reqtify as the requirements traceability solution.  It allows to create traceability directly to Word, thereby making traceability to Subset-26 easy.  However, there is no clear solution for authoring additional requirements (except using Word).  Further, it is not clear how traceability to model artifacts should be realized.  Last, this is a closed source solution and therefore only a last resort.
\item[Rodin and Pluggin.]
\item[Matelo.]
\item[Goal Structuring Notation (GSN).]
\item[Eclipse ProR.]
\item[Eclipse EMF Store.]
\item[Eclipse EMF Client Platform.]
\end{description}


During the evaluation phase, a number of challenges were identified that were not clearly defined before.
The list of challengers discussed during the evaluation is:

\begin{itemize}
\item Ecore model + XML  files
\item UML library
\item ReqCity
\end{itemize}


\subsection{ProR Evaluation}

TODO the details of the ProR evaluation, corresponding to the primary toolchain evaluations.


\subsection{EventB, Rodin and pluggins}

\begin{description}
\item[Name] Event-B and the Rodin platform
\item[Web site] \url{http://www.event-b.org}
\item[Licence] Common Public License Version 1.0 (CPL)
\end{description}

\paragraph{Abstract}

Rodin is an open source tool for formal modeling and verification on the system
level using the Event-B formalism. Event-B is based on set-theoretic notation of
first-order logic (FOL) and has its roots in the B method which has a long
history of successful application in industry on software level development.

Rodin is fully integrated into the Eclipse platform and is therefore fully
extensible through plug-ins. Existing plug-ins include graphical modeling using
state-machines, model simulators, modern state-of-the art SMT solvers and
Rational DOORS interoperable requirements tracing using ReqIf documents and
ProR.

\paragraph{Publications}

\begin{itemize}
\item The leaflet~\cite{RodinLeaflet} contains a short overview of the Rodin
  tool
\item The book~\cite{RodinHandbook} explains the usage of Rodin and serves as a
  gentle introduction into Event-B modeling in Rodin
\item The book~\cite{Abrial:2010:MES:1855020} contains an extensive presentation
  of Event-B an several modeling examples for different system
\item The scientific journal article~\cite{AbrialBHHMV10} contains an in-depth
  look at the integration of Event-B into the Rodin platform
\end{itemize}



A quantitative evaluation is available in \url{https://github.com/openETCS/toolchain/blob/master/T7.2/O7.2.1_Safety/O7-2-1_Safety.pdf}

\paragraph{Added value for OpenETCS project}


 Rodin is a specialized tool to formally model and verify abstract functional
  behavior. Therefore data management is not in its scope, as this is clearly a
  lower level detail aspect, more on the implementation level.

  \textbf{Function Management:} A Rodin model contains high level function
  descriptions, i.e.,\ an abstract view of the observable system behavior and
  its effect on the system state. It is therefore well suited to be included in
  function management, by formalizing the abstract behavior of the functions,
  tracing any changes and observing their effect on the intended functioning of
  the system.

  \textbf{Version Management:} Rodin does not contain a version management
  itself. Its files are based on XML, therefore any modern version control
  system can be used, in particular those (like svn/mercurial/git) for which an
  Eclipse plug-in exists. There also exists a pug-in that is compatible to
  model-compare in Eclipse, i.e.,\ allows for comparison on the model level
  instead of text level.

  \textbf{Other:} Rodin can provide an important support for
  \textbf{traceability}, which is missing here. It allows for linking formal
  model aspects to a requirements document, e.g.,\ a ReqIf document in ProR. Any
  changes in the specification can therefore be traced in the formal Event-B
  model and system-level aspects can be formally verified.

\paragraph{Integration in OpenETCS process and toolchain}

  The Rodin platform is fully based on Eclipse.

  The existing graphical modeling plug-ins for Rodin could be connected to
  Papyrus. This would require the development of a transformation of the
  different formats.

  With SCADE there could be the possibility of interoperation via the SCADE
  System SysML framework.

  With Classical B tools, there is the possibility to generate predicates for
  guards and invariants directly from the Event-B model. As classical B is based
  on text files and Event-B on XML file, there would be some development work to
  do.


%\section{Challenges}
%
%During the evaluation phase, a number of challenges were identified that were not clearly defined before.

\section{Open issues}

\subsection{Traceability}

TODO

\subsection{How to Deal with Subset-26}

TODO

%\subsection{Other Challenges}
%
%The list of challengers discussed during the evaluation is:
%
%\begin{itemize}
%\item Ecore model + XML  files
%\item UML library
%\item ReqCity
%\end{itemize}

\section{Selected means and tools}

\begin{comment}
To complete after decision meeting with a section for each tool with the following contents:

\begin{itemize}
\item description of the means or tools, references and links
\item added value for openETCS
\item for which tasks and how (input/output/actions) is the mean or tools used.
\end{itemize}
\end{comment}