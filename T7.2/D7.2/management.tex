

\chapter{Data and Requirements Management}
\label{sec:management}

 
This section is dedicated to tools and means to support management of data, functions requirements and other artifacts along  the openETCS process.

In total, seven tools have been proposed.  Out of these, only one has been evaluated in detail (ProR).  What follows is a qualitative description of the seven tools.  A quantitative evaluation of ProR is included as well.

\section{Candidates}
The list of initial candidates is:

\begin{description}
\item[Scade Suite.]  Scade includes Reqtify as the requirements traceability solution.  It allows to create traceability directly to Word, thereby making traceability to Subset-26 easy.  However, there is no clear solution for authoring additional requirements (except using Word).  Further, it is not clear how traceability to model artifacts should be realized.  Last, this is a closed source solution and therefore only a last resort.
\item[Rodin and Pluggin.]
\item[Matelo.]
\item[Goal Structuring Notation (GSN).]
\item[Eclipse ProR.]
\item[Eclipse EMF Store.]
\item[Eclipse EMF Client Platform.]
\end{description}

\section{ProR Evaluation}

TODO the details of the ProR evaluation, corresponding to the primary toolchain evaluations.

\section{Challenges}

During the evaluation phase, a number of challenges were identified that were not clearly defined before.

\subsection{Traceability}

TODO

\subsection{How to Deal with Subset-26}

TODO

\subsection{Other Challenges}

The list of challengers discussed during the evaluation is:

\begin{itemize}
\item Ecore model + XML  files
\item UML library
\item ReqCity
\end{itemize}

\section{Selected means and tools}

\begin{comment}
To complete after decision meeting with a section for each tool with the following contents:

\begin{itemize}
\item description of the means or tools, references and links
\item added value for openETCS
\item for which tasks and how (input/output/actions) is the mean or tools used.
\end{itemize}
\end{comment}