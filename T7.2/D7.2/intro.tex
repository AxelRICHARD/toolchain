% Start here


\chapter{Introduction}
\label{sec:intro}

The aim of this document is to report the results of the evaluation of means and tools for the secondary toolchain, i.e. the means and tools which complete the primary tool chain dedicated to formal model and software design.

This evaluation task is part of work package WP7, task 2 "Secondary  tools analyses and recommendations". According to the results of WP2, especially the OpenETCS process and the
requirements on language and tools \citep{D2_6}, and the results of T7.1 on the primary toolchain \citep{D7.1},  the aim of this task is to determine the best candidates to complete and support the primary toolchain for the following activities:

\begin{itemize}
\item data, function and requirement management (SSRS, WP3 and WP4), in chapter \ref{sec:management};
\item verification and  validation (WP4), in chapter \ref{sec:VnV};
\item safety activities support (WP4), in chapter \ref{sec:safety};
\item model transformation and code generation (WP3 and WP4), in chapter \ref{sec:transfo}.
\end{itemize}

\section{Approach}

Initially, it was planned to perform the secondary toolchain analysis in the same way the primary toolchain analysis was performed. However, it turned out not to be practical. Most importantly, team members were considering the secondary tools tactical (while the primary tools were strategic). Therefore, it was less clear what exactly was needed, and the impact of tool choice would be much more limited. Take model transformation for instance: The framework technology (EMF) was already given. It was clear that transformation would have to take place, but not yet which ones. Thus, it was unclear, which EMF model transformation framework would be the best one. Further, the choice would have little impact on the overall system (as the input and output of the model transformation would always be EMF models). Therefore, there was little motivation for an exhaustive evaluation.

As a result, this document contains the evaluated technologies, but only broadly describes those that have been evaluated or presented.


%%%%%%%%%%%%%%%%%%%%%%%%%%%%%%%%%%%%%%%%%%%%%%%%%%%%%%%%%%%%%%%



