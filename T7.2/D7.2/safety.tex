

\chapter{Safety support}
\label{sec:safety}

This section is dedicated to tools and means to support safety analyses.


\section{Candidates}



The list of initial candidates is:

\begin{itemize}
\item Rodin and Pluggins
\item CPN tools
\item Goal Structuring Notation (GSN)
\item Safety Architect
\end{itemize}

\subsection{EventB, Rodin and pluggins}

\begin{description}
\item[Name] Event-B and the Rodin platform
\item[Web site] \url{http://www.event-b.org}
\item[Licence] Common Public License Version 1.0 (CPL)
\end{description}

\paragraph{Abstract}

Rodin is an open source tool for formal modeling and verification on the system
level using the Event-B formalism. Event-B is based on set-theoretic notation of
first-order logic (FOL) and has its roots in the B method which has a long
history of successful application in industry on software level development.

Rodin is fully integrated into the Eclipse platform and is therefore fully
extensible through plug-ins. Existing plug-ins include graphical modeling using
state-machines, model simulators, modern state-of-the art SMT solvers and
Rational DOORS interoperable requirements tracing using ReqIf documents and
ProR.

\paragraph{Publications}

\begin{itemize}
\item The leaflet~\cite{RodinLeaflet} contains a short overview of the Rodin
  tool
\item The book~\cite{RodinHandbook} explains the usage of Rodin and serves as a
  gentle introduction into Event-B modeling in Rodin
\item The book~\cite{Abrial:2010:MES:1855020} contains an extensive presentation
  of Event-B an several modeling examples for different system
\item The scientific journal article~\cite{AbrialBHHMV10} contains an in-depth
  look at the integration of Event-B into the Rodin platform
\end{itemize}



A quantitative evaluation is available in \url{https://github.com/openETCS/toolchain/blob/master/T7.2/O7.2.1_Safety/O7-2-1_Safety.pdf}

\paragraph{Added value for OpenETCS project}

  Rodin and Event-B do not directly support a hazard or risk analysis. Their
  goal is to strengthen the confidence in the correctness of an external safety
  analysis, by providing means to represent safety requirements (in particular
  functional requirements) in a formal model and to verify them there or to
  validate the intended behavior wrt.\ safety by simulating and observing the
  model.

  Sub-system requirements can be specified and verified, if the formal model
  contains a representation of the sub-systems. While this can be achieved by
  refinement, it should be kept in mind that Event-B aims at system-level
  modeling and analysis, and therefore there could be better alternatives to
  analyze a very detailed model on implementation level.

  The main application of Rodin is to formalize and verify the safety
  requirements where applicable. This supports the verification of the
  correctness of the arguments in the safety case, therefore strengthening the
  confidence in these arguments, but also to provide insight into probably
  lacking aspects of the safety case.

 The Event-B approach is based on iterative refinements form the most abstract
  model to the desired level of detail. It is therefore a to-down approach, a
  bottom-up approach does not make sense using Event-B.

  And database connection would require the development of additional plug-ins,
  but would be possible.

  VnV of safety requirements is achieved by formal proof and simulation to
  validate correct functionality.

  Traceability is achieved by the connection to ProR.

  Generation of some documentation is already supported, as Latex documents can
  be generated from models. For more extensive documentation, e.g.,\ links with
  safety requirements, some additional functionality would have to be developed.
\

\paragraph{Integration in OpenETCS process and toolchain}

  The Rodin platform is fully based on Eclipse.

  The existing graphical modeling plug-ins for Rodin could be connected to
  Papyrus. This would require the development of a transformation of the
  different formats.

  With SCADE there could be the possibility of interoperation via the SCADE
  System SysML framework.

  With Classical B tools, there is the possibility to generate predicates for
  guards and invariants directly from the Event-B model. As classical B is based
  on text files and Event-B on XML file, there would be some development work to
  do.



\section{Selected means and tools}

\begin{comment}
To complete after decision meeting with a section for each tool with the following contents:

\begin{itemize}
\item description of the means or tools, references and links
\item added value for openETCS
\item for which tasks and how (input/output/actions) is the mean or tools used.
\end{itemize}
\end{comment}