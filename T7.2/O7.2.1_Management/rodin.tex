\chapter{Rodin}
\label{sec:rodin}


Partial evaluation.
Slides available on gothub \url{https://github.com/openETCS/model-evaluation/blob/master/Telco_Secondary_slides/Systerel_Event-B.pdf}.

\section{Author and Assessors}

\begin{description}
\item[\textcolor{green}{Author}] Matthias Güdemann --- Systerel
\item[\textcolor{blue}{Assessor 1}] First assessor of the approaches \todo{Name - Company}
\item[\textcolor{magenta}{Assessor 2}] Second assessor of the approaches \todo{Name - Company}
\end{description}

\section{Presentation}

\begin{description}
\item[Name] Event-B and the Rodin platform
\item[Web site] \url{http://www.event-b.org}
\item[Licence] Common Public License Version 1.0 (CPL)
\end{description}

\paragraph{Abstract}

Rodin is an open source tool for formal modeling and verification on the system
level using the Event-B formalism. Event-B is based on set-theoretic notation of
first-order logic (FOL) and has its roots in the B method which has a long
history of successful application in industry on software level development.

Rodin is fully integrated into the Eclipse platform and is therefore fully
extensible through plug-ins. Existing plug-ins include graphical modeling using
state-machines, model simulators, modern state-of-the art SMT solvers and
Rational DOORS interoperable requirements tracing using ReqIf documents and
ProR.

\paragraph{Publications}

\begin{itemize}
\item The leaflet~\cite{RodinLeaflet} contains a short overview of the Rodin
  tool
\item The book~\cite{RodinHandbook} explains the usage of Rodin and serves as a
  gentle introduction into Event-B modeling in Rodin
\item The book~\cite{Abrial:2010:MES:1855020} contains an extensive presentation
  of Event-B an several modeling examples for different system
\item The scientific journal article~\cite{AbrialBHHMV10} contains an in-depth
  look at the integration of Event-B into the Rodin platform
\end{itemize}

For which activities are dedicaded the means or tools (give a note from 0 to  3) :

\begin{tabular}{|l | c | c | c | c|}
\hline
& \textcolor{green}{Author} & \textcolor{blue}{Assessor 1} & \textcolor{magenta}{Assessor 2} & Total \\
\hline 
Data Management & 0  & & &  \\
\hline
Function Management & 2 & & & \\
\hline
Requirement Management & 2 & & & \\
\hline
Version Management & 2 & & & \\
\hline
Other (give details below) & 3 & & & \\
\hline
\end{tabular}

\begin{author_comment}
  Rodin is a specialized tool to formally model and verify abstract functional
  behavior. Therefore data management is not in its scope, as this is clearly a
  lower level detail aspect, more on the implementation level.

  \textbf{Function Management:} A Rodin model contains high level function
  descriptions, i.e.,\ an abstract view of the observable system behavior and
  its effect on the system state. It is therefore well suited to be included in
  function management, by formalizing the abstract behavior of the functions,
  tracing any changes and observing their effect on the intended functioning of
  the system.

  \textbf{Version Management:} Rodin does not contain a version management
  itself. Its files are based on XML, therefore any modern version control
  system can be used, in particular those (like svn/mercurial/git) for which an
  Eclipse plug-in exists. There also exists a pug-in that is compatible to
  model-compare in Eclipse, i.e.,\ allows for comparison on the model level
  instead of text level.

  \textbf{Other:} Rodin can provide an important support for
  \textbf{traceability}, which is missing here. It allows for linking formal
  model aspects to a requirements document, e.g.,\ a ReqIf document in ProR. Any
  changes in the specification can therefore be traced in the formal Event-B
  model and system-level aspects can be formally verified.
\end{author_comment}

\section{Common criteria on secondary means and tools}
\label{common}
This section discusses the common criteria of the means and tools according to the project requirements on tools and the results of T7.1.

\subsection{Project and WP2 requirements}

The objectives of this list of criteria is to check if the proposed means and tools meet the main criteria of the project: open-source approaches, usability, modularity, coverage of the objectives,...

According WP2 requirements, give a note for characteristics of the use of the tool (from 0 to 3) :

\begin{tabular}{|l | c | c | c | c|}
\hline
& \textcolor{green}{Author} & \textcolor{blue}{Assessor 1} & \textcolor{magenta}{Assessor 2} & Total \\
\hline 
Open Source (D2.6-02-074) & 3 & & &  \\
\hline 
Portability to operating systems (D2.6-02-075) & 3 & & &  \\
\hline
Cooperation of tools (D2.6-02-076) & 3 & & &  \\
\hline
Robustness (D2.6-02-078) & 3 & & & \\
\hline
Modularity (D2.6-02-078.1) & 3 & & & \\
\hline
Documentation management (D2.6-02-078.02) & 2 & & & \\
\hline
Distributed software development (D2.6-02-078.03)  & 3 & & & \\
\hline
Simultaneous multi-users (D2.6-02-078.04)   & 2 & & & \\
\hline
Issue tracking (D2.6-02-078.05) & 2 & & & \\
\hline
Differences between models (D2.6-02-078.06) & 2 & & & \\
\hline
Version management (D2.6-02-078.07) & 3 & & & \\
\hline
Concurrent version development (D2.6-02-078.08) & 3 & & & \\
\hline
Model-based version control (D2.6-02-078.09) & 2 & & & \\
\hline
Role traceability (D2.6-02-078.10) & 1 & & & \\
\hline
Safety version traceability (D2.6-02-078.11) & 3 & & & \\
\hline
Model traceability (D2.6-02-079) & 3 & & & \\
\hline
Tool chain integration & 3 & & & \\
\hline
Scalability & 2 & & & \\
\hline
User Friendliness & 2 & & & \\
\hline
\end{tabular}

\begin{author_comment}
  Rodin is based on Eclipse, therefore existing plug-ins can be used for many of
  the above aspects. Many of those are applicable without any changes, for
  others, some Rodin / Event-B specific modifications might be necessary.
\end{author_comment}

\subsection{Qualification}

This section discusses how the tool can be classified according EN50128 requirements (D2.6-02-085). Some qualification shall be mandatory  if the tool is involved to design a SIL4 software.


\begin{tabular}{|l | c | c | c | c|}
\hline
& \textcolor{green}{Author} & \textcolor{blue}{Assessor 1} & \textcolor{magenta}{Assessor 2} & Total \\
\hline 
Tool manual (D.2.6-01-42.02) & 3 & & &  \\
\hline
Proof of correctness (D.2.6-01-42.03)   & 2 & & & \\
\hline
Existing industrial  usage  & 3 & & & \\
\hline
Model verification & 3 & & & \\
\hline
Test generation & 0 & & & \\
\hline
Simulation, execution, debugging & 3 & & & \\
\hline
Formal proof & 3 & & & \\
\hline
\end{tabular}


Which level of tool qualification has been reached or will be reached within the next year ?

Score :
\begin{description}
\item[3] already qualified for this level
\item[2] qualification possible to this level, but some elements shall be provided
\item[0] qualification not recommended for this level
\end{description}


\begin{tabular}{|l | c | c | c | c|}
\hline
& \textcolor{green}{Author} & \textcolor{blue}{Assessor 1} & \textcolor{magenta}{Assessor 2} & Total \\
\hline 
class T1 & 2 & & &  \\
\hline
class T2  & 2 & & & \\
\hline
class T3  & 0 & & & \\
\hline
\end{tabular}

\begin{author_comment}
  The Rodin tool aims at system-level analysis, therefore it will not be
  necessary to qualify it as T3 tool, as no output is generated that can
  directly contribute to the executable code.
\end{author_comment}


\paragraph{Other elements for tool certification}


\section{Complementarity with primary toolchain}

The objectives of this list of criteria is to check if the proposed means and tools can be easily integrated to the primary toolchain.

\subsection{Language}


According to the decisions and the propositions of T7.1, how the mean and approach can be adapted to or can complete the chosen language and methods:

\begin{tabular}{|l | c | c | c | c|}
\hline
& \textcolor{green}{Author} & \textcolor{blue}{Assessor 1} & \textcolor{magenta}{Assessor 2} & Total \\
\hline 
SysML  & 2 & & & \\
\hline
Scade method & 1 & & & \\
\hline
EFS language & 0 & & & \\
\hline
B Method & 3 & & & \\
\hline
C language & 2 & & & \\
\hline
\end{tabular}


\paragraph{SysML}
How the means or tools can complete SysML ?

\begin{author_comment}
  Rodin allows graphical modeling of (UML) state machines, which are encoded
  into Event-B models. SysML state machines are very similar to this and with a
  bit of effort could be supported directly.
\end{author_comment}


\paragraph{Scade, EFS, Classical B}
How the means or tools can complete the current proposals for formal modeling language ?

\begin{author_comment}
  A light-weight interoperability with SCADE is possible, either via SCADE
  Systems which uses SysML or via SCADE state machines. This would allow a
  larger effort for integration. The data-flow part of SCADE does not seem to be
  applicable in an Event-B model.

  As Event-B has its roots in the B language, several aspects of these languages
  are definitively compatible. For example the invariant predicates of Event-B
  can directly be used in a lower level B model. If the abstraction levels for
  data are not the same, an additional refinement step could be added to solve
  this problem.

  There does not seem to be a good interoperation possibility with the EFS
  language.
\end{author_comment}


\paragraph{C language}
How the means or tools can complete or be adapted to SIL4 software in C language ?

\begin{author_comment}
  A possible combination of an Event-B model and a C implementation is to use
  the predicate logic invariants as C asserts and the guards as preconditions of
  functions. As the abstraction level of the C implementation is much lower than
  the Event-B models, this would require some work to identify the right
  functions and data formats or to introduce higher level wrapper functions
  similar to Event-B events. Such asserts and pre-conditions could be verified
  by tools like SPARK, why3 etc.
\end{author_comment}

\subsection{Tools and platforms}

According to the decisions and the propositions of T7.1, how the mean and approach can be integrated to or can complete the chosen tools and platforms:

\begin{tabular}{|l | c | c | c | c|}
\hline
& \textcolor{green}{Author} & \textcolor{blue}{Assessor 1} & \textcolor{magenta}{Assessor 2} & Total \\
\hline 
Eclipse & 3 & & &  \\
\hline
Papyrus  & 2 & & & \\
\hline
Scade & 1 & & & \\
\hline
EFS tools & 1 & & & \\
\hline
B tools & 2 & & & \\
\hline
\end{tabular}


\paragraph{Eclipse}
How the means or tools can be integrated to the Eclipse platform ?

\begin{author_comment}
  The Rodin platform is fully based on Eclipse.
\end{author_comment}

\paragraph{Papyrus}
How the means or tools can complete  Papyrus ?

\begin{author_comment}
  The existing graphical modeling plug-ins for Rodin could be connected to
  Papyrus. This would require the development of a transformation of the
  different formats.
\end{author_comment}


\paragraph{Scade, EFS, Classical B}
How the means or tools can complete the current proposals for formal modeling tools ?

\begin{author_comment}
  With SCADE there could be the possibility of interoperation via the SCADE
  System SysML framework.

  With Classical B tools, there is the possibility to generate predicates for
  guards and invariants directly from the Event-B model. As classical B is based
  on text files and Event-B on XML file, there would be some development work to
  do.

  For the EFS tools there are some interoperation possibilities on the EMF
  level, as both Rodin and EFS have an EMF model of the artifacts. However, as
  seen in the section above, how the two languages could interoperate is not
  clear.
\end{author_comment}


\section{Means and tools for data, function and requirement management}
\label{sec:management}


This section defines the criteria for the means and tools dedicated to data, function and requirement management. These activities are shared by the work packages WP3, WP4 and the activities dedicated to  SSRS.
These means and tools shall integrate the primary toolchain to  complete its gap and facilitate the integration of different activities. First of all, they  allow the management of a common repository of data, functions and requirements, shared between the models (from SSRS informal specification to code) and the verification and validation activities.  
Then, they shall support traceability of requirements between models and activities, and facilitate the verification of the traceability.
Besides they shall support the design of SIL4 software with model comparison or document production facilities, and version management.

\subsection{Management activities}

Which activites, linked to help the management of SSRS definition and whole process are covered by the mean or tool  ?

\begin{tabular}{|l | c | c | c | c|}
\hline
& \textcolor{green}{Author} & \textcolor{blue}{Assessor 1} & \textcolor{magenta}{Assessor 2} & Total \\
\hline 
Requirement capturing & 3 & & &  \\
\hline
Requirement management  & 2 & & & \\
\hline
Data management & 0 & & & \\
\hline
Function management & 2 & & & \\
\hline
Requirement traceability  & 3 & & & \\
\hline
Model traceability & 3 & & & \\
\hline
Function architecture & 2 & & & \\
\hline
Version management & 2 & & & \\
\hline
Model comparison & 2 & & & \\
\hline
Documentation production & 1 & & & \\
\hline
Others (give details) & & & & \\
\hline
\end{tabular}

\begin{author_comment}
  The requirements managing in Rodin is always done in close interoperation with
  ProR and one or more ReqIf documents. Via this plug-in, requirements can be
  linked directly to Event-B model artifacts, changes in either the model or the
  requirement document are marked for review.

  The formalization of the requirements in Rodin allows for a much more accurate
  analysis of consistency and correctness. This can be particular helpful at
  times of changes to the requirements, which can be very difficult to assess
  without tool support.

  Up to now, the document production integrated into Rodin is limited to the
  generation of a Latex representation of the model. This could be extended to
  also generate the information of elements linked to requirements in the
  requirements document.
\end{author_comment}


\subsection{Input Artifacts}

Which artifacts are used as input of the mean or tool  ? 


\begin{tabular}{|l | c | c | c | c|}
\hline
& \textcolor{green}{Author} & \textcolor{blue}{Assessor 1} & \textcolor{magenta}{Assessor 2} & Total \\
\hline 
Informal description & 1 & & &  \\
\hline
Structured description & 1 & & & \\
\hline
Spread sheet & 1 & & & \\
\hline
XML files & 3 & & & \\
\hline
EFS model & 1 & & & \\
\hline
DSL & 1 & & & \\
\hline
Others (give details) & & & & \\
\hline
\end{tabular}

\begin{author_comment}
  Rodin uses XML files as input means. All other formats would require the
  development of input filters.
\end{author_comment}

\subsection{Output Artifacts}

Which artifacts are used as output of the mean or tool  ? 


\begin{tabular}{|l | c | c | c | c|}
\hline
& \textcolor{green}{Author} & \textcolor{blue}{Assessor 1} & \textcolor{magenta}{Assessor 2} & Total \\
\hline 
Informal description & 0 & & &  \\
\hline
Structured description & 2 & & & \\
\hline
Spread sheet & 0 & & & \\
\hline
XML files & 2 & & & \\
\hline
EFS model & 0 & & & \\
\hline
DSL & 0 & & & \\
\hline
Others (give details) & 3 & & & \\
\hline
\end{tabular}

\begin{author_comment}
  Rodin produces documentation output of its models in Latex format and formal
  proof trees are saved in XML files. In general, its main output is a
  \textbf{formal description} of the predicates and proofs of the formal
  invariants for the system level functional model.
\end{author_comment}

\subsection{Requirement Management}

This section is link to reauirement definition and management activities.

Are these criteria coverd by the tool or mean ? 


\begin{tabular}{|l | c | c | c | c|}
\hline
& \textcolor{green}{Author} & \textcolor{blue}{Assessor 1} & \textcolor{magenta}{Assessor 2} & Total \\
\hline 
Editing of Textual Requirements & 3 & & &  \\
\hline
Represent Relations between Req (Textual-Based) & 3 & & & \\
\hline
Represent Relations between Req (Modelling-Based) & 3 & & & \\
\hline
Glossary and Abbreviation handling (Linked to Req)) & 2& & & \\
\hline
Traceability of Textual Requirements to Modelling & 3 & & & \\
\hline
Import/Export of Industrial Standard Data (e.g., REQIF) & 2 & & & \\
\hline
Documentation generation & 1 & & &  \\
\hline
Search and Filtering functions & 1 & & & \\
\hline
Others (give details) & & & & \\
\hline
\end{tabular}

\begin{author_comment}
  Rodin supports requirements managing with the ProR plug-in. Via its EMF model,
  its artifacts can be linked to elements of a ReqIf file via drag and
  drop. This allows for good traceability, as every change can be marked for
  later validation.

  Textual requirements tracing is also possible by using comments in the Event-B
  model. Search and filtering is partially supported by the Event-B editor, a
  better integration, more focused on requirements traceability could be
  developed with some effort.
\end{author_comment}

\section{Other comments}



\begin{author_comment}
  In general, the Rodin platform can give more confidence in the completeness
  and correctness of the requirements on the system level. The formalization
  allows to identify contradictions and missing elements in the specification.
\end{author_comment}


%%% Local Variables: 
%%% mode: latex
%%% TeX-master: "O7-2-1_Management"
%%% End: 
