\chapter{Acceleo}
\label{sec:acceleo}

\section{Instructions}

\begin{description}
\item[\textcolor{green}{Author}] Stefan Rieger (TWT)
\item[\textcolor{blue}{Assessor 1}] First assessor of the approaches \todo{Name - Company}
\item[\textcolor{magenta}{Assessor 2}] Second assessor of the approaches \todo{Name - Company}
\end{description}

In the sequel, main text is under the responsibilities of the author.

\begin{author_comment}
Author can add comments using this format at any place.
\end{author_comment}

\begin{assessor1}
First assessor can add comments using this format at any place.
\end{assessor1}

\begin{assessor2}
Second assessor can add comments using this format at any place.
\end{assessor2}


When a note is required, please follow this list (inspired from Technology Readiness Level, see http://en.wikipedia.org/wiki/Technology\_readiness\_level) :

\begin{description}
\item[0] not recommended / rejected / no integration possible or valuable / not adapted for this topic / not available for this topic
\item[1] weakly recommended / adapted after major improvements / weakly rejected / concept of integration roughly defined / adapted after major improvements / available after major developments
\item[2] recommended / adapted (with light improvements if necessary)  weakly accepted / integration prototyped or defined in details / adapted after small improvements / available after small developments or tests
\item[3] highly recommended / well adapted / strongly accepted / integration done and tested / well adapted to the purpose / available and suitable for the purpose All the notes can be commented under each table.
\item[*] difficult to evaluate with a note (please add a comment under the table)
\end{description}

All the notes can be commented under each table.

\section{Presentation}

This section gives a quick presentation of the approach and the tool.

\begin{description}
\item[Name] Acceleo
\item[Web site] http://www.eclipse.org/acceleo/
\item[Licence] Eclipse
\end{description}

\paragraph{Abstract} Short abstract on the approach and tool (10 lines max)
Acceleo is an implementation of the Object Management Group (OMG) MOF Model to Text Language (MTL) standard. Based on a special template language model to text transformations can be defined. It is fully integrated with Eclipse and also part of Polarsys.

\paragraph{Publications} Short list of publications on the approach (5 max)
Most imformation is available on the homepage http://www.eclipse.org/acceleo/

\section{Common criteria on secondary means and tools}
\label{common}
This section discusses the common criteria of the means and tools according to the project requirements on tools and the results of T7.1.

\subsection{Project and WP2 requirements}

The objectives of this list of criteria is to check if the proposed means and tools meet the main criteria of the project: open-source approaches, usability, modularity, coverage of the objectives,...

According WP2 requirements, give a note for characteristics of the use of the tool (from 0 to 3) :

\begin{tabular}{|l | c | c | c | c|}
\hline
& \textcolor{green}{Author} & \textcolor{blue}{Assessor 1} & \textcolor{magenta}{Assessor 2} & Total \\
\hline 
Open Source (D2.6-02-074) &3 & & &  \\
\hline 
Portability to operating systems (D2.6-02-075) &3 & & &  \\
\hline
Cooperation of tools (D2.6-02-076) &3 & & &  \\
\hline
Robustness (D2.6-02-078) &3* & & & \\
\hline
Modularity (D2.6-02-078.1) &3 & & & \\
\hline
Documentation management (D2.6-02-078.02) &0** & & & \\
\hline
Distributed software development (D2.6-02-078.03)  &0** & & & \\
\hline
Simultaneous multi-users (D2.6-02-078.04)   &0** & & & \\
\hline
Issue tracking (D2.6-02-078.05) &0** & & & \\
\hline
Differences between models (D2.6-02-078.06) &0** & & & \\
\hline
Version management (D2.6-02-078.07) &0** & & & \\
\hline
Concurrent version development (D2.6-02-078.08) &0** & & & \\
\hline
Model-based version control (D2.6-02-078.09) &0** & & & \\
\hline
Role traceability (D2.6-02-078.10) &0** & & & \\
\hline
Safety version traceability (D2.6-02-078.11) &0** & & & \\
\hline
Model traceability (D2.6-02-079) &0** & & & \\
\hline
Tool chain integration &3 & & & \\
\hline
Scalability &*** & & & \\
\hline
User Friendliness &3& & & \\
\hline
\end{tabular}

\begin{author_comment}
\begin{description}
\item[*] Sub-criteria of robustness in D2.6 do not make sense here, e.g., version management is not a sub-criterion to robustness.
\item[**] Out of scope of this tool. The requirements address an tool chain, so other tools should be used to cover these aspects.
\item[***] Scalability is difficult to judge and has not been evaluated.
\end{description}
\end{author_comment}

\subsection{Qualification}

This section discusses how the tool can be classified according EN50128 requirements (D2.6-02-085). Some qualification shall be mandatory  if the tool is involved to design a SIL4 software.


\begin{tabular}{|l | c | c | c | c|}
\hline
& \textcolor{green}{Author} & \textcolor{blue}{Assessor 1} & \textcolor{magenta}{Assessor 2} & Total \\
\hline 
Tool manual (D.2.6-01-42.02) &3 & & &  \\
\hline
Proof of correctness (D.2.6-01-42.03)   &0 & & & \\
\hline
Existing industrial  usage  &* & & & \\
\hline
Model verification &0 & & & \\
\hline
Test generation &0 & & & \\
\hline
Simulation, execution, debugging &0 & & & \\
\hline
Formal proof &0 & & & \\
\hline
\end{tabular}

\begin{author_comment}
The above table is not entirely clear to me. I filled the items 4-7 according to applicability of the tool.
\begin{description}
\item[*] Not checked in this context.
\end{description}
\end{author_comment}

Which level of tool qualification has been reached or will be reached within the next year ?

\begin{author_comment}
The possible answers below are not aligned with the above question and thus make no sense. This is an open source tool that is not / will not be pre-qualified by the tool community (as is, e.g., gcc). As the tool is open source a qualification should be possible but may involve considerable effort.
\end{author_comment}

Score :
\begin{description}
\item[3] already qualified for this level
\item[2] qualification possible to this level, but some elements shall be provided
\item[0] qualification not recommended for this level
\end{description}


\begin{tabular}{|l | c | c | c | c|}
\hline
& \textcolor{green}{Author} & \textcolor{blue}{Assessor 1} & \textcolor{magenta}{Assessor 2} & Total \\
\hline 
class T1 & & & &  \\
\hline
class T2   & & & & \\
\hline
class T3  & & & & \\
\hline
\end{tabular}

\paragraph{Other elements for tool certification}


\subsection{Complementarity with primary toolchain}

The objectives of this list of criteria is to check if the proposed means and tools can be easily integrated to the primary toolchain.

\subsubsection{Language}


According to the decisions and the propositions of T7.1, how the mean and approach can be adapted to or can complete the chosen language and methods:

\begin{tabular}{|l | c | c | c | c|}
\hline
& \textcolor{green}{Author} & \textcolor{blue}{Assessor 1} & \textcolor{magenta}{Assessor 2} & Total \\
\hline 
SysML  & 3& & & \\
\hline
Scade method & 2*& & & \\
\hline
EFS language & 2**& & & \\
\hline
B Method & 0& & & \\
\hline
C language & 3& & & \\
\hline
\end{tabular}

\begin{author_comment}
\begin{description}
\item[*] As Scade System is based on EMF and Papyrus an integration should be possible
\item[**] If based on EMF integration is possible
\end{description}
\end{author_comment}


\paragraph{SysML}
How the means or tools can complete SysML ?

\begin{author_comment}
Full integration with Eclipse and SysML is available.
\end{author_comment}


\paragraph{Scade, EFS, Classical B}
How the means or tools can complete the current proposals for formal modeling language ?

\begin{author_comment}
An integration with EMF-based tools should be possible.
\end{author_comment}

\paragraph{C language}
How the means or tools can complete or be adapted to SIL4 software in C language ?

\begin{author_comment}
C-Code (as code in any language) can be generated but generation templates need to be defined.
\end{author_comment}

\subsubsection{Tools and platforms}

According to the decisions and the propositions of T7.1, how the mean and approach can be integrated to or can complete the chosen tools and platforms:

\begin{author_comment}
This section in my opinion is redundant for Acceleo, see my answers above (the answers for Eclipse and Papyrus are the same as for SysML).
\end{author_comment}

\begin{tabular}{|l | c | c | c | c|}
\hline
& \textcolor{green}{Author} & \textcolor{blue}{Assessor 1} & \textcolor{magenta}{Assessor 2} & Total \\
\hline 
Eclipse & 3& & &  \\
\hline
Papyrus  & 3& & & \\
\hline
Scade & 2& & & \\
\hline
EFS tools & 2& & & \\
\hline
B tools &0 & & & \\
\hline
\end{tabular}


\paragraph{Eclipse}
How the means or tools can be integrated to the Eclipse platform ?

\begin{author_comment}
See comments regarding SysML above.
\end{author_comment}

\paragraph{Papyrus}
How the means or tools can complete  Papyrus ?

\begin{author_comment}
See comments regarding SysML above.
\end{author_comment}


\paragraph{Scade, EFS, Classical B}
How the means or tools can complete the current proposals for formal modeling tools ?

\begin{author_comment}
See comments regarding cade, EFS, Classical B above.
\end{author_comment}


\section{Means and tools for model transformation and code generation}
\label{sec:transformation}



This section defines the criteria for the means and tools dedicated to model and code transformation. These activities are shared by the work packages WP3 and WP4.


\subsection{Activities}

These transformations concern the design models  (from a model to an another, or to  executable code) but also validation activities (for model-based testing techniques for example).

Besides dedicated verification activities shall be necessary to  check these transformation (conformance, coverage, traceability,...)

Which transformations are covered by the mean or tool  ?

\begin{tabular}{|l | c | c | c | c|}
\hline
& \textcolor{green}{Author} & \textcolor{blue}{Assessor 1} & \textcolor{magenta}{Assessor 2} & Total \\
\hline 
Model transformation for design & 3& & &  \\
\hline
Model transformation for VnV  & 3& & & \\
\hline
Code Generation & 3& & & \\
\hline
\end{tabular}


\subsection{Input Artifacts}

Which artifacts are used as input of the mean or tool  ? 


\begin{tabular}{|l | c | c | c | c|}
\hline
& \textcolor{green}{Author} & \textcolor{blue}{Assessor 1} & \textcolor{magenta}{Assessor 2} & Total \\
\hline 
Informal description & & & &  \\
\hline
SysML model & 3& & & \\
\hline
Scade model & 2*& & & \\
\hline
EFS model & 2*& & & \\
\hline
Classical B modes & 0& & &  \\
\hline
C Code & 0& & & \\
\hline
Others (give details) & 0& & & \\
\hline
\end{tabular}

\begin{author_comment}
* EMF-based parts/tools
\end{author_comment}

\subsection{Output Artifacts}

Which artifacts are used as output of the mean or tool  ? 


\begin{tabular}{|l | c | c | c | c|}
\hline
& \textcolor{green}{Author} & \textcolor{blue}{Assessor 1} & \textcolor{magenta}{Assessor 2} & Total \\
\hline 
Informal description & 0& & &  \\
\hline
SysML model & 1*& & & \\
\hline
Scade model & 0& & & \\
\hline
EFS model & 0& & & \\
\hline
Classical B modes & 0& & &  \\
\hline
C Code & 3& & & \\
\hline
Others (give details) & 3**& & & \\
\hline
\end{tabular}

\begin{author_comment}
\begin{description}
\item[*] Model-to-model transformation is possible if the generated model can be imported as text. However SysML $\rightarrow$ SysML does not make much sense in my opinion.
\item[**] Code in any language or text-based files/models can be generated.
\end{description}
\end{author_comment}

\subsection{Process}


How process the tool, with which characteristics (please provides comments) ? 


\begin{tabular}{|l | c | c | c | c|}
\hline
& \textcolor{green}{Author} & \textcolor{blue}{Assessor 1} & \textcolor{magenta}{Assessor 2} & Total \\
\hline 
Informal & 0& & &  \\
\hline
Model To Text (M2T) & 3& & & \\
\hline
Model To Model (M2M) & 2*& & & \\
\hline
EMF models compliant & 3& & & \\
\hline
others & 3*& & & \\
\hline
\end{tabular}

\begin{author_comment}
\begin{description}
\item[*] Code in any language or text-based files/models can be generated.
\end{description}
\end{author_comment}

\section{Other comments}



\begin{comment}
This section is available for the author or the assessors to  complete the description and criteria.
\end{comment}



