% Start here


\chapter{Introduction}
\label{sec:intro}


The aim of this document is to report the results of the evaluation of means and tools for the secondary means and tools, i.e. the means and tools which complete the primary tool chain dedicated to formal model and software design.

This evaluation task is part of work package WP7, task 2 "Secondary  tools analyses and recommendations". According to the results of WP2, especially the OpenETCS process and the
requirements on language and tools \citep{D2_6}, and the results of T7.1 on the primary toolchain \citep{D7.1},  the aim of this task is to determine the best candidates to complete and support the primary toolchain for the following activities:

\begin{itemize}
\item verification and  validation
\item safety activities support
\item data, function and requirement management
\item model transformation and code generation
\end{itemize}

\section{Organisation of the document}

The chapter  \ref{sec:template} provides a template to describe the means and tools and a list
of criteria according WP2 requirements on language, models and tools, and T7.1 primary tool chain decision. The objectives of this
description and criteria are to allow to determine the best means of description and associated
tool for a given activities.


The chapter \ref{sec:concl} resumes the results of the evaluation at the end of the benchmark activities.

In Appendix, a chapter is dedicated to each models produced during the benchmark activities :

\begin{itemize}
\item Scade Suite
\item System C
\item Rodin and Pluggins
\item Tools around Classical B (ProB, SMT solver,...)
\item CPN tools
\item Matelo
\item RT-Tester
\item Fiacre and Tina
\item Gnat Prove
\item Frama-C
\item Diversity
\item Acceleo
\item ATL
\item QVTO and SmartQVT
\item Goal Structuring Notation (GSN)
\item Eclipse ProR
\item Safety Architect
\end{itemize}

%%%%%%%%%%%%%%%%%%%%%%%%%%%%%%%%%%%%%%%%%%%%%%%%%%%%%%%%%%%%%%%


\section{Glossary}
\label{sec:glossary}

\begin{description}
\item[API] Application Programming Interface
\item[FME(C)A] Failure Mode Effect (and Criticity) Analysis
\item[FIS] Functional Interface Specification
\item[HW] Hardware
\item[I/O] Input/Output
\item[OBU] On-Board Unit
\item[PHA] Preliminary Hazard Analysis
\item[QA] Quality Analysis
\item[RBC] Radio Block Center
\item[RTM] RunTime Model
\item[SIL] Safety Integrity Level
\item[SRS] System Requirement Specification
\item[SSHA] Sub-System Hazard Analysis
\item[SSRS] Sub-System Requirement Specification
\item[SW] Software
\item[THR] Tolerable Hazard Rate
\item[V\&V] Verification \& Validation
\end{description}

%%%%%%%%%%%%%%%%%%%%%%%%%%%%%%%%%%%%%%%%%%%%%%%%%%%%%%%%%%%%%%%


