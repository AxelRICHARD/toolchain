\chapter{GOPRR}

\begin{description}
\item[\textcolor{green}{Author}] Author of the approaches description
  \textcolor{green}{Johannes Feuser/C\'ecile Braunstein  (Uni. Bremen)}
\item[\textcolor{blue}{Assessor 1}] First assessor of the approaches \textcolor{blue}{Alexandre Ginisty (All4Tec)}
\item[\textcolor{magenta}{Assessor 2}] Second assessor of the approaches \textcolor{magenta}{Matthias G\"udemann (Systerel)}
\end{description}

In the sequel, main text is under the responsibilities of the author.

\begin{author_comment}
Author can add comments using this format at any place.
\end{author_comment}

\begin{assessor1}
First assessor can add comments using this format at any place.
\end{assessor1}

\begin{assessor2}
Second assessor can add comments using this format at any place.
\end{assessor2}

When a note is required, please follow this list :
\begin{description}
\item[0] not recommended, not adapted, rejected
\item[1] weakly recommended, adapted after major improvements, weakly rejected
\item[2] recommended, adapted (with light improvements if necessary)  weakly accepted
\item[3] highly recommended, well adapted,strongly accepted
\item[*] difficult to evaluate with a note (please add a comment under the table)
\end{description}

All the notes can be commented under each table.

\section{Presentation}

This section gives a quick presentation of the approach and the tool.

\begin{description}
\item[Name] An openETCS donain specific language Based on GOPPRR
\item[Web site] http://www.informatik.uni-bremen.de/agbs/jfeuser/
\item[Licence] GPL V3
\end{description}

\paragraph{Abstract} 
The approach proposes an open  domain specific language for
openETCS. The language is based on the meta-meta-model  GOPPRR, and the
tool used is MetaEdit+.


\paragraph{Publications} Short list of publications on the approach (5 max)
\begin{itemize}
\item ``MetaEdit+ Workbench User’s Guide'' accessed April 27, 2011. [Online].:
http://www.metacase.com/support/45/manuals/mwb/Mw.html
\item S. Kelly and J.-P. Tolvanen, ``Domain-Specific Modeling''. JOHN WILEY \& SONS, INC.,
2008.
\item J. Feuser and J. Peleska, ``Model Based Development and Tests for openETCS Applications
– A Comprehensive Tool Chain'', 12 2012, in in Proceedings of FORMS/FORMAT 2012.
\item J. Feuser, ``Open source Software for Train Control and
  Application and its Architectural Implication'', PhD. Thesis, 2013.
\end{itemize}


\section{Main usage of the approach}
\label{main_usage}
This section discusses the main usage of the approach.

According to the figure \ref{fig:main_process}, for which phases do you recommend the approach (give a note from 0 to  3) :

\begin{tabular}{|l | c | c | c | c|}
\hline
& \textcolor{green}{Author} & \textcolor{blue}{Assessor 1} & \textcolor{magenta}{Assessor 2} & Total \\
\hline 
System Analysis &
2 &2 & 1& 5 \\
\hline
Sub-system formal design &3 &3 & 3&  9 \\
\hline
Software design &3 &3 & 3& 9 \\
\hline
Software code generation &3 &3 & 3& 9 \\
\hline
\end{tabular}

According to the figure \ref{fig:main_process}, for which type of activities do you recommend the approach (give a note from 0 to  3) :

\begin{tabular}{|l | c | c | c | c|}
\hline
& \textcolor{green}{Author} & \textcolor{blue}{Assessor 1} & \textcolor{magenta}{Assessor 2} & Total \\
\hline 
Documentation &1 &1 & 1 & 3 \\
\hline
Modeling &3 &3 & 3&  9 \\
\hline
Design &2 &2 & 2 & 6 \\
\hline
Code generation &3 &3 & 3& 9 \\
\hline
Verification &0 &0 & 0& 0 \\
\hline
Validation &0 &0 & 0& 0 \\
\hline
Safety analysis &0 &0 & 0& 0 \\
\hline
\end{tabular}

\paragraph{Known usages} Have you some examples of usage of this approach to  compare with the OpenETCS objectives ?

\section{Language}
This section discusses the main element of the language.

Which are the main characteristics of the language :

\begin{tabular}{|l | c | c | c | c|}
  \hline
  & \textcolor{green}{Author} & \textcolor{blue}{Assessor 1} & \textcolor{magenta}{Assessor 2} & Total \\
  \hline 
  Informal language &0 &0 & 0& 0  \\
  \hline 
  Semi-formal language &0 &0 & 0&  0 \\
  \hline
  Formal language &3 &2 & 3& 8 \\
  \hline
  Structured language &3 &3 & 3& 9 \\
  \hline
  Modular language &3 &3 & 3& 9 \\
  \hline
  Textual language &0 &0 & 0& 0 \\
  \hline
  Mathematical symbols or code &0 &0 & 0& 0 \\
  \hline
  Graphical language &3 &3 & 3& 9 \\
  \hline
\end{tabular}

According WP2 requirements, give a note for the capabilities of the language (from 0 to 3) :

\begin{tabular}{|l | c | c | c | c|}
  \hline
  & \textcolor{green}{Author} & \textcolor{blue}{Assessor 1} & \textcolor{magenta}{Assessor 2} & Total \\
  \hline
  Declarative formalization of properties (D2.6-02-066) &3 &2 & 2& 7 \\
  \hline
  Simple formalization of properties (D2.6-02-066.01) &2 &2 & 2& 6 \\
  \hline
  Scalability : capability to design large model &3 &3 & 2& 8 \\
  \hline
  Easily translatable to other languages (D2.6-02-068) &3 &3 & 3& 9 \\
  \hline
  Executable directly (D2.6-02-071) &0 &0 & 0& 0 \\
  \hline
  Executable after translation to a code (D2.6-02-071) &3 &3 & 3& 9 \\
  (precise if the translation is automatic) &3 &2 & 3& 8 \\
  \hline
  Simulation, animation (D2.6-02-071) &0 &0 & 0& 0 \\
  \hline
  Easily understandable (D2.6-02-065) &2 &2 & 2& 6 \\
  \hline
  Expertise level needed (0 High level, 3 few level) &2 &2 & 2& 6 \\
  \hline
  Standardization (D2.6-02-067) &* &* & *& * \\
  \hline
  Documented (D2.6-02-067) &3 &2 & 2& 7 \\
  \hline
  Extensible language (D.2.6-01-28) &3 &3 & 2*& 8\\
  \hline
\end{tabular}

\textcolor{magenta}{Assessor 2} wrt. extensibility: the language is designed to
be specific for ETCS modeling.

\begin{author_comment}
(*) The meta-meta model (GOPPRR) is not a standard but it is formally defined.
\end{author_comment}

\paragraph{Documentation} Describe how the language is documented, the existing guidelines, coding rules, standardization...

The language is fully documented, and the meta-model of the language
is given.
\paragraph{Language usage} Describe the possible restriction on the language

\section{System Analysis}
This section discusses the usage of the approach for system analysis.
It can be skipped depending the results of \ref{main_usage}.

Acoording WP2 requirements, how the approach can be involved for the sub-system requirement specification ?

\begin{tabular}{|l | c | c | c | c|}
\hline
& \textcolor{green}{Author} & \textcolor{blue}{Assessor 1} & \textcolor{magenta}{Assessor 2} & Total \\
\hline
Independent System functions definition (D2.6-02-045.02.1)  &2 &2 & 2& 6 \\
\hline 
System architecture design (D2.6-02-045.02) &3 &3 & 3& 9 \\
\hline
System data flow identification (D2.6-02-045.02.3)  &3 &3 & 3& 9 \\
\hline
Sub-system focus (D2.6-02-045.02.4)  &3 &3 & 3& 9 \\
\hline
System interfaces definition (D2.6-02-045.02.5)  &3 &3 & 3& 9 \\
\hline
System requirement allocation (D2.6-02-045.03)  &3 &3 & 2&  8 \\
\hline
Traceability with SRS (D2.6-02-045.05)  &2 &2 & 2& 6  \\
\hline
Traceability with Safety activities (D2.6-02-046)  &2 &2 & 2 & 6   \\
\hline
\end{tabular}



\section{Sub-System formal design}
This section discusses the usage of the approach for sub-system formal design.
It can be skipped depending the results of \ref{main_usage}.

Two kinds of model can be planned during this phase: semi-formal models to  cover the SSRS (D2.6-02-047.01) and strictly formal  models to  focuss on some functional and safety aspects (D2.6-02-049).  Obviously some strictly  formal means can be used to define the semi-formal  model.

\subsection{Semi-formal model}

\begin{comment}
Section has been skipped.
\end{comment}

\subsection{Strictly formal model}

Concerning strictly formal model, how the WP2 requirements are covered ?

\begin{tabular}{|l | c | c | c | c|}
\hline
& \textcolor{green}{Author} & \textcolor{blue}{Assessor 1} & \textcolor{magenta}{Assessor 2} & Total \\
\hline 
Consistency to SFM (D2.6-02-049.2) &* &* & *&  * \\
\hline
Coverage of SSRS (D2.6-02-049.2)  &1 &1 & 1& 3  \\
\hline
Traceability to  SSRS (D2.6-02-049.3)  &3 &2 & 2& 7 \\
\hline
Extensible to software design (D2.6-02-051)  &1 &1 & 1& 3  \\
\hline
Safety function isolation (D2.6-02-052)  &1 &1 & 1& 3 \\
\hline 
Safety properties formalisation (D2.6-02-057)  &2 &2 & 1& 5 \\
\hline
Logical expression (D2.6-02-066.02.01)  &3 &3 & 3& 9 \\
\hline
Timing constraints (D2.6-02-066.02.02)  &0 &0 & 0& 0 \\
\hline
Safety properties validation (D2.6-02-058.03)  &3 &2 & 2& 7 \\
\hline
Logical properties assertion (D2.6-02-072)  &3 &3 & 3& 9 \\
\hline
Proof of assertions (D2.6-02-072.2)  &3 &3 & 2& 8 \\
\hline
\end{tabular}

\begin{author_comment}
(*) Since the semi-formal model is not yet done, it is hard to say.
\end{author_comment}
Does the language allow to  formalize (D2.6-02-070):

\begin{tabular}{|l | c | c | c | c|}
\hline
& \textcolor{green}{Author} & \textcolor{blue}{Assessor 1} & \textcolor{magenta}{Assessor 2} & Total \\
\hline 
State machines  &3 &3 & 3 & 9 \\
\hline
Time-outs  &0 &0 & 0& 0 \\
\hline
Truth tables  &0 &0 & 0& 0 \\
\hline
Arithmetic  &3 &3 & 3& 9 \\
\hline
Braking curves  &3 &3 & 3& 9 \\
\hline
Logical statements &3 &3 & 3& 9 \\
\hline
Message and fields &3 &3 & 3& 9 \\
\hline
\end{tabular}

\paragraph{Additional comments on semi-formal  model} Do you think your strictly formal  model can be directly defined from the SSRS ?
All comments on links to  other models, validation and verification activities are welcomed.


\section{Software design}
This section discusses the usage of the approach for software design.
It can be skipped depending the results of \ref{main_usage}.

\subsection{Functional design}

How the approach allows to  produce a functional software model of the on-board unit ?

\begin{tabular}{|l | c | c | c | c|}
\hline
& \textcolor{green}{Author} & \textcolor{blue}{Assessor 1} & \textcolor{magenta}{Assessor 2} & Total \\
\hline
Derivation from system semi-formal model  &* &* & *& * \\
\hline 
Software architecture description  &3 &3 & 3& 9 \\
\hline
Software constraints  &3 &3 & 3& 9 \\
\hline
Traceability  &3 &3 & 3& 9 \\
\hline
Executable  &3 &2 & 3& 8 \\
\hline
\end{tabular}
\begin{author_comment}
(*) Since we do not know the format of the semi-formal model  yet, it is hard to say.
\end{author_comment}

\subsection{SSIL4 design}

How the approach allows to  produce in safety a software model ?

\begin{tabular}{|l | c | c | c | c|}
\hline
& \textcolor{green}{Author} & \textcolor{blue}{Assessor 1} & \textcolor{magenta}{Assessor 2} & Total \\
\hline
Derivation from system semi-formal or strictly formal model  &* &* & *& * \\
\hline 
Software architecture description  &3 &3 & 3& 9 \\
\hline
Software constraints  &3 &3 & 3& 9 \\
\hline
Traceability  &3 &3 & 3& 9 \\
\hline
Executable  &3 &2 & 3& 8 \\
\hline
Conformance to EN50128 § 7.2  &0 &0 & 0& 0 \\
\hline
Conformance to EN50128 § 7.3  &0 &0 & 0& 0 \\
\hline
Conformance to EN50128 § 7.4  &0 &0 & 0& 0 \\
\hline
\end{tabular}
\begin{author_comment}
(*) Since we do not know the format of the semi-formal model  yet, it is hard to say.
\end{author_comment}

Which criteria for software architecture are covered by the methodology
(see EN50128 table A.3) :

\begin{tabular}{|l | c | c | c | c|}
\hline
& \textcolor{green}{Author} & \textcolor{blue}{Assessor 1} & \textcolor{magenta}{Assessor 2} & Total \\
\hline
Defensive programming  &3 &3 & 1*& 7* \\
\hline 
Fault detection \& diagnostic  &3 &3 & 1& 7 \\
\hline
Error detecting code  &3 &3 & 1& 7 \\
\hline
Failure assertion programming &3 &3 & 3& 9 \\
\hline
Diverse programming &1 &1 & 1& 3 \\
\hline
Memorising executed cases &0 &0 & 0& 0 \\
\hline
Software error effect analysis &0 &0 & 0& 0 \\
\hline
Fully defined interface &3 &3 & 3& 9 \\
\hline
Modeling  &3 &3 & 3& 9 \\
\hline
Structured methodology &3 &3 & 3& 9 \\
\hline
\end{tabular}

\textcolor{magenta}{Assessor 2} I am not aware of explicit means to defensive
programming, nor are they mentioned in Johannes' thesis.

\section{Software code generation}
This section discusses the usage of the approach for software code generation.
It can be skipped depending the results of \ref{main_usage}.

Which criteria for software design and implementation are covered by the methodology
(see EN50128 table A.4) :

\begin{tabular}{|l | c | c | c | c|}
\hline
& \textcolor{green}{Author} & \textcolor{blue}{Assessor 1} & \textcolor{magenta}{Assessor 2} & Total \\
\hline
Formal methods  &3 &3 & * & 6* \\
\hline 
Modeling  &3 &3 & * & 6* \\
\hline
Modular approach (mandatory) &3 &3 & 3& 9 \\
\hline
Components &3 &3 & 3& 9 \\
\hline
Design and coding standards (mandatory) &3 &3 & *& 6* \\
\hline
Strongly typed programming language &3 &3 & 3& 9 \\
\hline

\end{tabular}

\textcolor{magenta}{Assessor 2} Are there coding standards?

\section{Main usage of the tool}
\label{main_usage}

This section discusses the main usage of the tool.

Which task are covered by the tool ?


\begin{tabular}{|l | c | c | c | c|}
\hline
& \textcolor{green}{Author} & \textcolor{blue}{Assessor 1} & \textcolor{magenta}{Assessor 2} & Total \\
\hline 
Modelling support &3 &3 & 3& 9 \\
\hline
Automatic translation  &1 &1 & 1& 3 \\
\hline
Code Generation  &3 &3 & 3& 9 \\
\hline
Model verification &2 &2 & 2& 6 \\
\hline
Test generation &* &* & *& * \\
\hline
Simulation, execution, debugging &2 &2 & 2& 6 \\
\hline
Formal proof &0 &0 & 0& 0 \\
\hline
\end{tabular}
\begin{author_comment}
(*) The new version of MetaEdit+ provides this feature.
\end{author_comment}

\paragraph{Modeling support}
Does the tool provide a  textual or a graphical editor ?

Graphical
\paragraph{Automatic translation and code generation}
Which translation or code generation is supported by the tool ?

MERL generator
\paragraph{Model verification}
Which verification on models are provided by the tool?

Inspection

\paragraph{Test generation}
Does the tool allow to generate tests ? For  which purpose ?

No
\paragraph{Simulation, execution, debugging}
Does the tool allow to simulate or to debbug step by step a model or a code ?


The new version yes.
\paragraph{Formal proof}
Does the tool allow formal proof ?  How ?

No

\section{Use of the tool}


According WP2 requirements, give a note for characteristics of the use of the tool (from 0 to 3) :

\begin{tabular}{|l | c | c | c | c|}
\hline
& \textcolor{green}{Author} & \textcolor{blue}{Assessor 1} & \textcolor{magenta}{Assessor 2} & Total \\
\hline 
Open Source (D2.6-02-074) &0 &0 & 0& 0  \\
\hline 
Portability to operating systems (D2.6-02-075) &3 &3 & 3&  9 \\
\hline
Cooperation of tools (D2.6-02-076) &1 &1 & 1& 3 \\
\hline
Robustness (D2.6-02-078) &3 &3 & *& 6* \\
\hline
Modularity (D2.6-02-078.1) &0 &0 & 0& 0 \\
\hline
Documentation management (D2.6-02-078.02) &0 &0 & 0& 0\\
\hline
Distributed software development (D2.6-02-078.03)  &3 &3 & 3& 9\\
\hline
Simultaneous multi-users (D2.6-02-078.04)   &3 &3 & 3& 9 \\
\hline
Issue tracking (D2.6-02-078.05) &0 &0 & 0& 0 \\
\hline
Differences between models (D2.6-02-078.06) &0 &0 & 0& 0 \\
\hline
Version management (D2.6-02-078.07) &1 &1 & 1& 3 \\
\hline
Concurrent version development (D2.6-02-078.08) &0 &0 & 0& 0 \\
\hline
Model-based version control (D2.6-02-078.09) &0 &0 & 0& 0 \\
\hline
Role traceability (D2.6-02-078.10) &0 &0 & 0& 0 \\
\hline
Safety version traceability (D2.6-02-078.11) &0 &0 & 0& 0 \\
\hline
Model traceability (D2.6-02-079) &1 &1 & 1& 3 \\
\hline
Tool chain integration &3 &3 & *& 6* \\
\hline
Scalability &3 &3 & * & 6*\\
\hline
\end{tabular}

\textcolor{magenta}{Assessor 2} Could not test the robustness of Metaedit+.

Seems to be a stand-alone tool, so toolchain integration is difficult to assess
without trying the tool.

\section{Certifiability}

This section discusses how the tool can be classified according EN50128 requirements (D2.6-02-085).


\begin{tabular}{|l | c | c | c | c|}
\hline
& \textcolor{green}{Author} & \textcolor{blue}{Assessor 1} & \textcolor{magenta}{Assessor 2} & Total \\
\hline 
Tool manual (D.2.6-01-42.02) &3 &3 & 3& 9 \\
\hline
Proof of correctness (D.2.6-01-42.03)   &0 &0 & 0& 0 \\
\hline
Existing industrial  usage  &3 &3 & 3& 9 \\
\hline
Model verification &1 &1 & 1& 3 \\
\hline
Test generation &0 &0 & 0& 0 \\
\hline
Simulation, execution, debugging &0 &0 & 0& 0 \\
\hline
Formal proof &0 &0 & 0& 0 \\
\hline
\end{tabular}

\paragraph{Other elements for tool certification}

\section{Other comments}
Please to  give free comments on the approach.




%  LocalWords:  MetaEdit GOPPRR
