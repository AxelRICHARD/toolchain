\documentclass{beamer}

% This file is a solution template for:

% - Talk at a conference/colloquium.
% - Talk length is about 20min.


% Copyright 2004 by Till Tantau <tantau@users.sourceforge.net>.
%
% In principle, this file can be redistributed and/or modified under
% the terms of the GNU Public License, version 2.
%
% However, this file is supposed to be a template to be modified
% for your own needs. For this reason, if you use this file as a
% template and not specifically distribute it as part of a another
% package/program, I grant the extra permission to freely copy and
% modify this file as you see fit and even to delete this copyright
% notice. 


\mode<presentation>
{
  \usetheme{openETCS}
  % or ...

  \setbeamercovered{transparent}
  % or whatever (possibly just delete it)
  \setbeamertemplate{navigation symbols}{}

}


\usepackage[english]{babel}
% or whatever

\usepackage[utf8]{inputenc}
% or whatever

\usepackage{times}
\usepackage[T1]{fontenc}
% Or whatever. Note that the encoding and the font should match. If T1
% does not look nice, try deleting the line with the fontenc.


\title[Decision meeting T7.1] % (optional, use only with long paper titles)
{T7.1: Mainly follow-up on decisions}

\subtitle
{WP7 Review meeting Task 7.1}

\author[WP7 - Task 7.1] % (optional, use only with lots of authors)
{Marielle Petit-Doche}
% - Give the names in the same order as the appear in the paper.
% - Use the \inst{?} command only if the authors have different
%   affiliation.

\institute[Universities of Somewhere and Elsewhere] % (optional, but mostly needed)
{\pgfdeclareimage[height=1cm]{systerel-logo}{logoSysterel}
 \pgfuseimage{systerel-logo}
 }
% - Use the \inst command only if there are several affiliations.
% - Keep it simple, no one is interested in your street address.

\date[14th of January, Munich] % (optional, should be abbreviation of conference name)
{14th of January 2014, DB, Munich}
% - Either use conference name or its abbreviation.
% - Not really informative to the audience, more for people (including
%   yourself) who are reading the slides online

\subject{OpenETCS WP7-T7.1 Last Review Meeting}
% This is only inserted into the PDF information catalog. Can be left
% out. 


% If you have a file called "university-logo-filename.xxx", where xxx
% is a graphic format that can be processed by latex or pdflatex,
% resp., then you can add a logo as follows:

%\pgfdeclareimage[height=0.5cm]{university-logo}{logo-uni}
%\logoinst{\pgfuseimage{university-logo}}



% Delete this, if you do not want the table of contents to pop up at
% the beginning of each subsection:
\AtBeginSubsection[]
{
  \begin{frame}<beamer>{Outline}
    \tableofcontents[currentsection,currentsubsection]
  \end{frame}
}


% If you wish to uncover everything in a step-wise fashion, uncomment
% the following command: 

%\beamerdefaultoverlayspecification{<+->}


\begin{document}

\begin{frame}
  \titlepage
\end{frame}

\begin{frame}{Outline}
  \tableofcontents
  % You might wish to add the option [pausesections]
\end{frame}


% Structuring a talk is a difficult task and the following structure
% may not be suitable. Here are some rules that apply for this
% solution: 

% - Exactly two or three sections (other than the summary).
% - At *most* three subsections per section.
% - Talk about 30s to 2min per frame. So there should be between about
%   15 and 30 frames, all told.

% - A conference audience is likely to know very little of what you
%   are going to talk about. So *simplify*!
% - In a 20min talk, getting the main ideas across is hard
%   enough. Leave out details, even if it means being less precise than
%   you think necessary.
% - If you omit details that are vital to the proof/implementation,
%   just say so once. Everybody will be happy with that.



\section{SSRS needs}


\begin{frame}{Aim}

  \begin{itemize}
  \item
    Identify the functional architecture of the system (graphical view ?)
  \item
    Manage a set of function
  \item 
  	Manage a repository of requirements
  \item 
  	Manage a data dictionnary
  \end{itemize}

\end{frame}

\section{Expected inputs}

\begin{frame}{Next step}

  \begin{itemize}
  \item
    Specify information needed to describe function
  \item
    Specify information needed to  describe requirement
  \item 
  	Specify information needed to describe data
  \item 
  	Specify links between function, requirement, data
  \end{itemize}
  
  \textcolor{red}{$ \Rightarrow $ Urgent} to have a precise specification of the minimal needs:  Next week

\end{frame}


\section{WP7-T7.2 Concerns}


\begin{frame}{Requirements on means to support activities}

  \begin{itemize}
  \item
    Possibilities to share between users
  \item
    Possibilities to share information between activities
  \item 
  	Flexibility : possibility to add new elements to describes the items
  \item
  Open source
  \item 
  	\dots
  \end{itemize}
  
\end{frame}



\begin{frame}{Initial solutions ?}

  \begin{itemize}
  \item
    Pen and paper
  \item
    Excel sheet
  \item 
  	SysML
  \item
  	Plug-ins on eclipse (ProR,...)
  \item 
  	\dots
  \end{itemize}
  
\end{frame}


\section{Organisation}



\begin{frame}{T7.2 Organisation}

\begin{description}
\item[Inputs] Data model needed by end of July
\item[Benchmark] Analyse of needs and possible solution of support
\item[Decision] With other secondary tools :  end of September
\end{description}
  
\end{frame}



\section{T7.1- Sum-up of the Decisions}

\begin{frame}{T7.1 Results}
 \textcolor{blue}{\textbf{October 2013 Braunschweig PCC meeting}}
  \begin{itemize}
  \item D7.1 document accepted
  \item
	12 Decisions proposed, 11 accepted by vote     
  \item Action Plan agreed: 
  \begin{itemize}
  \item start the openETCS Charta (ERTMSSolutions, CEA, NS, EclipseSource, Alstom)
  \item prepare the toolchain decision
  \end{itemize} 
  \item WP3 activities can start
  \item WP7 development activities can start
  \end{itemize}
 
\end{frame}

\begin{frame}{T7.1 On-going (1)}

\begin{description}
\item[Decision 1] By vote in Paris on July 4th, 2013, all the partners agree on the use of Eclipse as tool platform.
\item[On-going] WP7 toolchain development on Eclipse, secndary tool selection in accordance
\end{description}
\end{frame}


\begin{frame}{T7.1 On-going (2)}

\begin{description}
\item[Decision 2] By vote in Paris on July 4th, 2013, all the partners agree on the use Papyrus/SysML to cover the highest level of the OpenETCS V-cycle. Later it was clarified that it will be an original or bidirectional synchronized (but not a derived) artifact.
\item[Discussion]  NS/LLRE did not agree to the proposal. The difference in opinion can be seen as a different interpretation of the process.
\item[Errata] added in D7.1
\item[On-going] Papyrus included in the toolchain (WP7), SysML modelling started (WP3)
\end{description}
\end{frame}

\begin{frame}{T7.1 On-going (3)}

\begin{description}
\item[Decision 3] As the first tool chain release, WP7 will provide at a minimum a Papyrus SysML environment to WP3, under the assumption that this is sufficient for WP3 to start modeling, until the final decision has been made (see Decision 5).
\item[On-going] First release provided (WP7), modelling started (WP3 task force)
\end{description}
\end{frame}


\begin{frame}{T7.1 On-going (4)}

\begin{description}
\item[Decision 7] Unless new evidence is provided in agreement with Decision 5, SCADE will be considered the foundation for the primary tool chain. In that case, a feasible migration from what had been done with SCADE is essential.
\item[On-going] Modelling started with SCADE (WP3 task force)
\item[Open Issue] Migration ?
\end{description}
\end{frame}


%
\section{T7.1 - Follow-up on Decisions}


\begin{frame}{T7.1 Open Issues (1)}

\begin{description}
\item[Decision 5] A final decision on the primary tool chain must be made by the end of January 2014. After this point, WP7 resources must not be spread between competing tools.
\item[Discussion]  an action plan is currently not visible to prepare for this decision and how we are going to proceed with this decision in January. Foundation of the plan has to be the openETCS Charta which needs to be started now to be available in January. The base for the charta is seen in the Eclipse Way of Life.
\item[Feedbacks] from the partners ?
\end{description}

\end{frame}

\begin{frame}{T7.1 Open Issues (2)}

\begin{description}
\item[Decision 8] Work on the tool chain must always take the objective of an OSS migration into account, e.g. by standardizing interfaces between tools. It is acknowledged that such a migration may not be feasible within the timeframe of the openETCS project. However, by the end of the project, feasibility of a migration must be demonstrated (e.g. with a prototype or case study).
\item[Discussion]  Rephrasing the text was recommended in order to increase the understanding of the document.
\item[Feedbacks] from the partners ?
\end{description}

\end{frame}



\begin{frame}{T7.1 Open Issues (3)}

\begin{description}
\item[Decision 9] All openETCS project partners must be made aware of the need for acquiring a SCADE license (depending on their activities). If a partner feels that this is not achievable, this must be escalated to the project office by November 15, 2013. The project office will assist in finding a solution. If no solution has been found by January 31, 2014 for all partners who escalated, then SCADE will be abandoned for anything on the critical path of the tool chain.
\item[Feedbacks] from the partners ?

\end{description}

\end{frame}

\begin{frame}{T7.1 Open Issues (4)}

\begin{description}
\item[Decision 10] According to the spirit of an open proofs project, team members are permitted to continue evaluating B, even with the objective of contradicting Decision 7. However, if unsuccessful by the deadline set by Decision 5, they must refocus either by contributing elsewhere, or by focusing on the transition to OSS, as outlined in Figure 57. The team is strongly encouraged to coordinate with POLARSYS.
\item[Discussion]  The proposal is seen to special. A more general recommendation is requested.
\item[Feedbacks] from the partners ?
\end{description}

\end{frame}


%
%
%
%** PCC Project Coordination Meeting Minutes**
%
%    Meeting called by Klaus-Rüdiger Hase
%    When: Tuesday, 8th of October, 13:00 - 16:00
%    Where: Braunschweig, DLR-Location
%    Minutes taken by Bernd Hekele
%    Participants:
%        AEbt: Merlin Pokam
%        All4Tec: Cyril Cornu
%        Alstom: Pierre-Francois Jauquet
%        CEA: Matthieu Perin, Virgile Prevosto
%        DB: Klaus-Rüdiger Hase, Baseliyos Jacob, Bernd Hekele
%        DLR: Marc Behrens, Bernd Gonska
%        EclipseSource: Jonas Helming
%        ERSA: Patrick Deutsch
%        ERTMS-Solutions: Stan Pinte
%        FormalMind: Michael Jastram
%        Fraunhofer: Jens Gerlach, Alexander Stante
%        General Electric: Giovanni Zanelli
%        IT-Telecom: Ana Cavalli
%        LAAS-CNRS: Silvano Dal Zilio
%        Lloyds: Jan Welvaarts
%        Mitsubishi Electric MERCE: David Mentré
%        NS-Rail: Jos Holtzer
%        Siemens: Gerhard Assmann, Uwe Steinke
%        SQS: Izaskun de la Torre
%        Systerel: Marielle Petit-Doche
%        TWT: Stefan Rieger
%        TU-Braunschweig: Jan Welte
%        Uni-Bremen: Cecile Braunstein
%        Uni Rostock: Alexander Nitsch
%
%Topics:
%
%    Welcome address / Objectives for this PCC Meeting
%    Wrap-up of ITEA2 review and recent activities in WP3 + WP7
%    Proposals for further actions
%    Decisions
%
%Findings:
%
%Topics 1., 2. and 3. are covered by slides 1 - 13 of the presentation
%4. Decisions
%
%It was intended to vote on the decisions as a whole lot. However, in order to get an view on the acceptance of the individual decisions a walk thru the Decisions 3. - 12. was performed. The result of the decisions is documented in the rear part of this document.
%
%After having seen some objections against parts of the proposal, the following conclusion was agreed on:
%
%    The document D7.1 was accepted as a document providing a good working base for the next steps. The goal of the document was accepted.
%
%    In order to work on the gaps discussed during the PCC an Action Plan was agreed. This Action Plan has the target
%        to start the openETCS Charta. Participants for the work group: ERTMSSolutions, CEA, NS, EclipseSource, Alstom
%        prepare the toolchain decision
%
%In the subsequent WP3 meeting(Braunschweg, 9th of October 2013) the start of WP3 owned System Analysis was confirmed. It was agreed to start a task force initiated by Alstom and with participation of all major players committing to contribute to the SysML Analysis with reasonable effort (Alstom, Siemens, DB, NS/LLRE and Systerel). The first workshop is scheduled for October 22nd - 25th)
%Results of walk thru the decisions:
%
%    Decision 1: By vote in Paris on July 4th, 2013, all the partners agree on the use of Eclipse as tool platform.
%        Result: accepted, no objections, no abstentions
%
%    Decision 2: By vote in Paris on July 4th, 2013, all the partners agree on the use Papyrus/SysML to cover the highest level of the OpenETCS V-cycle. Later it was clarified that it will be an original or bidirectional synchronized (but not a derived) artifact.
%        Result: accepted, 1 objections, no abstentions
%
%NS/LLRE did not agree to the proposal. The difference in opinion can be seen as a different interpretation of the process.
%
%    Decision 3: With input from WP2 and WP3, a WP7 will develop a validator for Papyrus-based SysML.
%        No objections, Abstention by 5 Partners (ERTMSSolutions, EclipseSource, Alstom, All4Tec, Systerel). Rest agreed.
%
%    Decision 4: As the first tool chain release, WP7 will provide at a minimum a Papyrus SysML environment to WP3, under the assumption that this is suficient for WP3 to start modeling, until the final decision has been made (see Decision 5).
%        Result: accepted, no objections, 1 abstention (ERTMSSolutions)
%
%    Decision 5: A final decision on the primary tool chain must be made by the end of January 2014. After this point, WP7 resources must not be spread between competing tools.
%
%        Discussion: an action plan is currently not visible to prepare for this decision and how we are going to proceed with this decision in January. Foundation of the plan has to be the openETCS Charta which needs to be started now to be available in January. The base for the charta is seen in the Eclips Way of Life.
%
%        Result: accepted, no objections, 2 abstention (ERTMSSolutions, Alstom)
%
%    Decision 6: ERTMSSolutions activities will be coordinated with WP4, to support V&V activities (as these are modeling activities, they still belong in WP3). This issue must be raised and resolved with the PCC.
%        Discussion: With this shift of resources resource allocation substitution of efforts by other partners is needed. The Doodle initiated with this query showed substantial indication by DB, Systerel and Siemens to cooperate in WP3.
%
%However, the approach was not supported by a majority because of the missing charta in openETCS.
%
%    Result: not accepted, 2 objections, 17 abstention
%
%    Decision 7: Unless new evidence is provided in agreement with Decision 5, SCADE will be considered the foundation for the primary tool chain. In that case, a feasible migration from what had been done with SCADE is essential.
%
%        Discussion: the term "evidence" is seen ambiguous and shall be removed.
%
%        Result: accepted, no objections, 5 abstention (ERTMSSolutions, EclipsSource, MERCE, Systerel, Alstom).
%
%    Decision 8: Work on the tool chain must always take the objective of an OSS migration into account, e.g. by standardizing interfaces between tools. It is acknowledged that such a migration may not be feasible within the timeframe of the openETCS project. However, by the end of the project, feasibility of a migration must be demonstrated (e.g. with a prototype or case study).
%
%        Discussion: Rephrasing the text was recommended in order to increase the understanding of the document.
%
%        Result: accepted, no objections, 2 abstention (ERTMSSolutions, Alstom)
%
%    Decision 9: All openETCS project partners must be made aware of the need for acquiring a SCADE license (depending on their activities). If a partner feels that this is not achievable, this must be escalated to the project office by November 15, 2013. The project office will assist in finding a solution. If no solution has been found by January 31, 2014 for all partners who escalated, then SCADE will be abandoned for anything on the critical path of the tool chain.
%        Result: accepted, no objections, 3 abstention (ERTMSSolutions, AEbt, Alstom)
%
%    Decision 10: According to the spirit of an open proofs project, team members are permitted to continue evaluating B, even with the objective of contradicting Decision 7. However, if unsuccessful by the deadline set by Decision 5, they must refocus either by contributing elsewhere, or by focusing on the transition to OSS, as outlined in Figure 57. The team is strongly encouraged to coordinate with POLARSYS.
%        Discussion: The proposal is seen to special. A more general recommendation is requested.
%        Result: accepted, 2 objections (LLRE, Alstom) + 5 abstentions.
%
%    Decision 11: WP4 is going to consider the artifacts of "SysML+ SCADE" tool chain for the first round of verification and validation launched in July 2013.
%        Discussion: The proposal is seen to special. A more general recommendation is requested.
%        Result: accepted, no objections + 1 abstention (ERTMSSolutions).
%
%    Decision 12: For the following rounds, starting in February 2014, WP4 will adapt the verification and validation activities to the models and code provides by WP3.
%
%    Decision on D7.1 as a complete package:
%        Result: accepted, no objections, 5 abstention (ERTMSSolutions, NS, LLRE, Systerel, Alstom).
%
% 



\section{T7.1 Conclusion}


\begin{frame}{T7.1 Proposition}



\begin{itemize}
\item More than a year for the evaluation (1/3 of project duration)
\item Now at the middle term of the project
\item Two decision meetings with votes
\item Few activities mentioned as follow-up of the meetings
\item Next steps started in the project (WP7, WP3, WP4,...)
\item Few resources available
\end{itemize}

  \pause
  \textcolor{red}{Can we consider officially this task achieved ?}

\end{frame}



%\input{old_contents}


\end{document}


