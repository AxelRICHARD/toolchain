\begin{description}
\item[Name] Eclipse and the Eclipse Modeling Framework (EMF)
\item[Web site] http://eclipse.org http://eclipse.org/emf
\item[License] EPL
\end{description}


\section{Abstract}
Eclipse is an open source Tool Platform originally developed at IBM. It has been explicitly designed as an extensible platform to enable different tools to exchange data and share common functionality. Additionally Eclipse is a rich open source ecosystem with a variety of frameworks for different purposes, such as versioning, code generation, language support and many more. The Eclipse Modeling Framework (EMF) as a top level project bundles all modeling frameworks at Eclipse. Additionally it technically provides a common data format for modeling purposes. Originally it has implemented the OMG Standard Meta-Object Facility (MOF) and has then be reduce to the OMG standard essential MOF (eMOF). EMF provides model-driven approach to develop modeling languages. It allows to define custom meta-models and generate code form them. Additionally it provides common features such as command-based editing and XMI serialization for generated models. In the following we show how Eclipse and EMF aligns with the openETCS requirements.
\section{Open Source}
All Eclipse core components including EMF are open source and under the Eclipse Public licenses, which allows for commercial use and is compatible to the EUPL. The Eclipse Foundation and the Eclipse Development process assure the management of the intellectual properties for all Eclipse projects. Additionally all Eclipse projects follow a common infrastructure and process allowing external partners to contribute and maintain projects. 
\section{Long-Term Maintenance}
The Eclipse Foundation also provides infrastructure and a process for Long-Term Maintenance for all Eclipse projects. It enables users of a technology to contract service providers to maintain current and older versions of these technologies. These service providers do not necessarily have to be committers on the original projects.
\section{Portability}
Eclipse itself is implemented in Java and therefore portable to all major operating systems. The underlying UI technology SWT is implemented for all major and even most uncommon window kits. As SWT uses native widgets, the performance of the UI is close to native applications. The Eclipse Java IDE has a user based of several million developers, which ensures, that the platform runs stable on the supported platforms. Since version 4.2, EMF is part of the core platform. However, EMF does not contain any OS specific components and is therefore highly portable.
\section{Tools Interoperability}
The Eclipse Platform has been explicitly designed to enable various tools of the software lifecycle to collaborate. It provides mechanisms, such as a service oriented architecture and extension points to enable the communication between different parts of a tool chain. EMF is well-suited as a common data-format. The collaboration of a large number of tools is shown and validated in the various Eclipse packages, which are released in the yearly release train.
\section{Modularity}
Eclipse is based on OSGi, a standard for modularization of Java applications. The Eclipse OSGi runtime Equinox is the reference implementation of OSGi. OSGi enables to modularize a system, in this case the tool chain. Additionally it allows to specify the API of modules and the dependencies between them. Additionally, the existing platform provides many possibilities to be extended by new features. The extensibility and OSGi as an underlying technology allow fully customizing the Eclipse Platform. Existing pieces and frameworks can be added to a tool chain, new parts can be developed.
\section{Framework Support}
Over the last ten years, a rich ecosystem of frameworks has developed around the Eclipse Platform. All these frameworks are developed under the EPL and checked for IP cleanliness. Eclipse frameworks cover all different kinds of purposes, however there is a strong focus in support for tool development and modeling. Modeling technologies are almost all compatible with EMF as a common data format. Technologies provided by Eclipse projects include:
\begin{enumerate}
\item Textual Modeling and DSL (e.g. Xtext)
\item Language Support (e.g. CDT, JDT)
\item Source Code Versioning Clients (e.g. Egit, Subclipse, Subversion)
\item Model Repositories and Versioning (e.g. EMFCompare, EMF Diff/merge, EMFStore and CDO)
\item Code Generation (e.g. Xpand, Xtend)
\item Model Transformation (e.g. ATL, QVT)
\item Model Development Tools (e.g. Papyrus, OCL, RMF, Sphinx, eTrice)
\item Graphical Modeling (e.g. Graphiti, GMF)
\item User Interfaces (e.g. JFace, Databinding, EMF Client Platform, EEF)
\item ALM Tooling (e.g. Mylyn)
\end{enumerate}
