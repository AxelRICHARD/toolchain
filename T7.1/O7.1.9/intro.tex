\chapter{Introduction}
This document is the results report of the evaluation of the tool
platform. This work is the first part of the tool platform selection
as described in the WP7 description of work \cite{WP7_D01}.
This task aims at evaluating tool platform integration capabilities,
regardless the primary and second tools that will be chosen for the
tool chain implementation.

The tool platform should provide mechanisms to integrate various
tools. The tool platform is not the primary nor secondary tools, nor
the tool chain. It is the support for the tool chain implementation,
it shall help to integrate the tools into a seamless tool chain.
The evaluation will focus on the integration capabilities of the tool platform.

This document will set up a way to evaluate tool platform and to
determine the set of best candidate according to the WP2 requirements
\cite{baro_requirements_2013}. This task will be followed by finding
the best tool platform candidates, e.g.finding the best the adequacy
between primary and secondary tools and the tool platform.


\section{The tool integration problem}
The tool platform raises the question of tool integration.
Wassermann \cite{wasserman_tool_1990}  defines the 5 problems of tool integration that must
be addressed:
\begin{enumerate}
\item Platform integration: How to deal with heterogeneous OS ?
\item Data integration: How to share data and how to manage their
  relationships ?
\item Presentation integration: How to unify the user-interface, the ``look and feel'' ?
\item Control integration: How to enable service sharing between tools and
how to notify one another of events ?
\item Process integration: How to support engineering process within
  the tool platform ?
\end{enumerate}
More recently  Asplund and al. in \cite{asplund_tool_2011} add some
metrics to evaluate the tool chain integration. Following these
propositions {\bf the tool platform evaluation will focus on
these integration capabilities}. 

Moreover, one of the openETCS's goal is to develop a tool chain to
generate a software meeting the CENELEC EN 50128:2011 requirements and
certifiable \gls{SIL}4. {\bf The tool platform shall help to support the EN 50128
standard}. 

The system integraty level of the product depends on the process
defined by the tool chain. As shown by Asplund and al. in
\cite{asplund_qualifying_2012}, sometimes reasoning tool by tool is
not suffisient to analyze the resulted product.  This implies that the
tool chain should be considered as a whole and not only individual
tools. Moreover, Asplund and al. as well as Slotoch and al. in
\cite{slotosch_iso_2012} shows that taking a holistic approach and use
some rearrangement or extension of the tool chain allow to avoid the
qualification of all tools and mitigate the certification process.
The tool chain doest not have to be certified but considering the tool
chain as whole will help to identify which part should be qualified in
order to obtain a certifiable software.  The evaluation of {\bf The
  tool platform with regards to the tool chain analysis capabilities}
will be made.


\section{Structure of the document}

Chapter \ref{sec:template} presents the template of the evaluation
criteria.
The initial candidates where the following
The appendix show the evaluation of each tool platform :
\begin{itemize}
\item Eclipse
\item SCADE
\item TopCased/Polarsys
\item MONO
\item RTP-Cesar
\end{itemize}

Nevertheless the only tool that can be considered as a tool platform
is Eclipse, thus only this tool has been evaluated.

%------ List of terms and definition ----------------
\printglossary


%%% Local Variables: 
%%% mode: latex
%%% TeX-master: "Evaluation_Platform_against_WP2"
%%% End: 
